% cv.tex
%—————————————————————————————————––
% Simple asymmetric two-column CV using vars.tex for overrides
%—————————————————————————————————––
\documentclass[a4paper,10pt]{article}
\usepackage[vmargin=1.5cm, hmargin=1.5cm]{geometry}

% Load general setup (fonts, spacing, tikz, adjustbox, etc.)
\input{MySetup}

% Load branch-specific variables
% vars.tex
% ———————––
% Overrideable variables for CV

%––––––––––––––––––––
% Personal details
\newcommand{\FullName}{Laura W. Dijkhuizen}
%\newcommand{\BirthDate}{}
%\newcommand{\BirthPlace}{}
\newcommand{\Residence}{Utrecht, the Netherlands}

\newcommand{\EmailAddress}{laura.w.dijkhuizen@pm.me}
\newcommand{\WebPage}{https://lauradijkhuizen.com}

%––––––––––––––––––––
% File paths
\newcommand{\ProfilePhoto}{pf.jpg}
\newcommand{\BackgroundImage}{bg-image.pdf}% e.g. a faint watermark or logo

%––––––––––––––––––––
% Style colors (hex RGB without #)
\definecolor{ColorOne}{HTML}{003399}   % Primary accent color
\definecolor{ColorTwo}{HTML}{FF6600}   % Secondary accent color

%––––––––––––––––––––
% Icon definitions (if needed)
% \newcommand{\IconEmail}{\faEnvelopeO}
% \newcommand{\IconLink}{\faChain}

%––––––––––––––––––––
% Any other branch-specific overrides go here

%—————————————————————————————————––
\begin{document}
\pagestyle{fancy}
\thispagestyle{empty}

% setup column nr 1 —————————————————————————————————––
\noindent\makebox[\textwidth][s]{
\begin{adjustbox}{valign=t}
\noindent\begin{minipage}[t]{0.3\textwidth}

\begin{center}
  \begin{tikzpicture}
      \clip (0,0) ellipse (2cm and 2.5cm);
      \node {\includegraphics[width=4cm]{\ProfilePhoto}};
  \end{tikzpicture}

{\begin{spacing}{.8}
  \LARGE \bfseries \FullName
\end{spacing}}

\MySkip

\fontfamily{pag}\selectfont
  \ifdefined\BirthDate Born on \BirthDate\\ \fi
  \ifdefined\BirthPlace \BirthPlace\\ \fi
  \ifdefined\Residence Currently living in \Residence\\ \fi

\MySkip
\fontfamily{pag}\selectfont
\textcolor{ForestGreen}{\faEnvelope} \myhref{mailto:\EmailAddress}{\EmailAddress}\\
\textcolor{ForestGreen}{\faGlobe} \myhref{\WebPage}{\WebPage}
\end{center}

\vfill

% About me
{\fontfamily{pag}\selectfont\RaggedRight
  Passionate about plant biology and making complex science accessible.
  I bring a strong background in plant biology, bioinformatics, and team coordination. 
  Skilled communicator thriving in autonomous, mission-driven teams.
}

% Professional Attributes
{\fontfamily{pag}\selectfont\RaggedRight

\GreenHeading{Professional Attributes}
\begin{spacing}{1}
\begin{itemize}
  \setlength{\itemindent}{-1em}
  \setlength{\itemsep}{0.0em}
  \item Clear \textbf{communicator} across technical \& non-technical audiences
  \item Skilled \textbf{educator \& mentor} to 400+ young professionals
  \item Experienced \textbf{chair} \& consensus builder representing $\sim$2000 PhDs
  \item \textbf{Autonomous multitasker} with project leadership
  \item \textbf{Genomics data scientist} (R, Python, Bash/Linux)
  \item \textbf{HPC \& workflow specialist}: Docker, GitHub, reproducible pipelines
  \item \textbf{Domain specialist} on plant symbiosis biology, genomics \& environmental research 
\end{itemize}
\end{spacing}
}

% Close column 1 put in vertical line as column2 —————————————————————————————————––

\end{minipage}%
\end{adjustbox}%
\hfill%


\begin{adjustbox}{valign=t}
\hfill%
\begin{minipage}[t]{0.05\textwidth}
\MyVerticalRule
\end{minipage}%
\end{adjustbox}
% Close column 2 and setup column nr 3 —————————————————————————————————––

\begin{adjustbox}{valign=t}
\hfill%
\begin{minipage}[t]{0.6\textwidth}

% Contents column 3 —————————————————————————————————––

% Employment
\GreenHeading{Employment}
\begin{description}
\raggedright
\item[\normalfont \textcolor{ForestGreen}{\textbf{2022 -- now.}}] \textbf{Lecturer \&  Trainer} in programming, bioinformatics \& data science for PhD candidates \& postdocs \\
\textit{Theoretical Biology \& Bioinformatics Group at Utrecht University}
\item[\normalfont \textcolor{ForestGreen}{\textbf{2017 -- 2022.}}] \textbf{PhD Researcher and teacher} I secured funding for my own PhD on the (meta)genomics of novel crop \textit{Azolla} \\ 
\textit{Molecular Plant Physiology Group, Utrecht University.}
\end{description}
% Detailed career note: While lab leader was abroad, led the “Azolla” research team—supervising students and guiding an interdisciplinary group to meet project milestones.
% (Original Azolla chapter; see portfolio at \myhref{https://lauradijkhuizen.com}{lauradijkhuizen.com})

% Management
\GreenHeading{Leadership \& Project Management}
\begin{description}
  \raggedright
\item[\normalfont \textcolor{ForestGreen}{\textbf{2023 -- now.}}] Member, \textbf{Department Advisory Committee}, Biology Dept., Utrecht University.
  -- feedback on strategic priorities \& governance.
\item[\normalfont \textcolor{ForestGreen}{\textbf{2021.}}] Member, TU Delft \textbf{Research Assessment Committee} (Chemical Engineering \& Biotechnology).
  -- "Visitatie commissie" in Dutch.
\item[\normalfont \textcolor{ForestGreen}{\textbf{2020 -- 2021.}}] Member, \textbf{Faculty Open Science Implementation Team} \& \textbf{UU Open Science Platform}.
  -- driving reproducibility policies and innovations.
  \item[\normalfont \textcolor{ForestGreen}{\textbf{2019 -- 2021.}}] Member, \textbf{Curriculum Committee}, M.Sc. Bioinformatics\& Biocomplexity.
  -- Design a brand new Masters curriculum.
  \item[\normalfont \textcolor{ForestGreen}{\textbf{2017 -- 2021.}}] \textbf{PhD Council Chair} and member \textbf{Board of Studies} (GS-LS) — represenging $\sim$2000 PhDs.
  -- negotiated new PhD graduation guidelines with three faculties.
\item[\normalfont \textcolor{ForestGreen}{\textbf{2017 -- 2021.}}] \textbf{PhD Council Representative}, Institute of Environmental Biology.
\end{description}
% Hint: occasional leadership of Azolla R&D team; guided interdisciplinary labs and committees while mentoring junior researchers.

% Key Achievements & Metrics
\GreenHeading{Key Achievements}
\begin{itemize}
    \setlength{\itemindent}{0em}
    \setlength{\parindent}{5em}
  \item Trained over 400 PhD candidates \& postdocs in my self made bioinformatics curriculum for professionals.
  \item Taught “Intro to Bioinformatics” course; cohesively uniting various domain experts from Utrecht Science Park.
  \item Skilled builder of reproducible pipelines like Nanopore variant calling, phylogeny workflow, metagenome analysis.
  \item Chaired a council representing $\sim$2000 PhD candidates, leading organisational change for new graduation guidelines adopted by three faculties.
  \item Experienced stakeholder manager and communicator serving on 5 more strategic committees.
  %\item Interim led a small lab during long-term leadership absence.
  \item Experienced public speaker from elementary schools to scientific conferences and of course classroom.
  %\item Featured in many media outlets (national newspaper, radio \& local TV), bringing complex science to the public.
  \item Build and maintained a bioinformatics Linux server enabling experimentalist colleagues to use my infrastructure with relative ease.
  \item Worked with a wide variety of data. From PacBio long reads to microRNA-seq and from optical mapping to handwritten notes.
\end{itemize}


% Contents page 2 —————————————————————————————————––
\end{minipage}%
\end{adjustbox}%
}
\newpage

% Education
\GreenHeading{Education}
\begin{description}
  \raggedright
  \item[\normalfont \textcolor{ForestGreen}{\textbf{2022.}}] Basic University Teaching Qualification (\textbf{bUTQ}), Utrecht University
  \item[\normalfont \textcolor{ForestGreen}{\textbf{2017 -- 2022.}}] \textbf{PhD: The Azolla metagenome} Utrecht University – Molecular Plant Physiology
  \item[\normalfont \textcolor{ForestGreen}{\textbf{2010 -- 2017.}}] \textbf{MSc / BSc} Utrecht University - Environmental Plant Biology
\end{description}

% Teaching & Training
\GreenHeading{Teaching \& Training}
\begin{description}
  \raggedright
  \item \textbf{Lecturer \& Trainer (2022–now)}: Design and deliver modular workshops in R, Python \& Bash for PhD candidates and postdocs, tailoring content to individual research goals and team projects.
  \item \textbf{Course Coordinator (2020–2021)}: Led the “Introduction to Bioinformatics” Master’s course—integrating 8+ domain experts from across Utrecht Science Park into a unified, hands-on curriculum (\href{https://github.com/lauralwd/metagenomicspractical}{GitHub}).
  %\item \textbf{Instructor, MSc Courses (2018–2021)}: Taught “Biology of the Bio-Based Economy” and “Introduction to Bioinformatics,” emphasizing applied data analysis, visualization, and reproducibility.
  \item \textbf{Thesis Supervisor (2017–2022)}: Mentored 10+ MSc/BSc students on computational biology and plant-symbiosis projects—guiding experimental design, data pipelines, and results communication.
  \item \textbf{Workshop Facilitator \& TA (2017–2020)}: Supported courses in Systems Biology, Plant Physiology, Molecular Genetics Techniques, and specialized bioinformatics bootcamps like Git.
  %\item \textbf{Educational Innovator}: Developed Docker-based teaching environments and collaborative online coding exercises.
  % HPC playground
\end{description}


% Outreach & Communication
\GreenHeading{Outreach \& Communication}
\begin{description}
  \raggedright
  \item \textbf{Television:} Local TV interview on \textit{Azolla} ferns (2017; Dutch) — \myhref{https://youtu.be/OI4VV4M2-f4}{Watch}. 
  And an interview for “De Kennis van Nu” popular science program (2018; Dutch) — \myhref{https://www.npostart.nl/focus/07-12-2018/VPWON_1296556}{Watch}.
  \item \textbf{Radio:} BNR national radio feature on Azolla (2017; Dutch) — \myhref{https://www.bnr.nl/podcast/wetenschap-vandaag/10346708/utrechts-plantje-geniet-wereldwijde-faam}{Link}
  \item \textbf{Print:} Feature in AD newspaper (2018; Dutch) -- \myhref{https://www.ad.nl/utrecht/kroosachtig-plantje-uit-sloot-naast-galgenwaard-blijkt-ware-eiwitbom~a1eaba6d/}{Read}
%  \item \textbf{Online:} UU News interview on Open Science views (2021; EN/NL) — \myhref{https://www.uu.nl/en/news/meet-laura-dijkhuizen}{Read}
  \item \textbf{Lectures \& Events:} Frequent invited speaker and demonstrator at public science events, including gene editing seminars, hands-on plant biology demos, and outreach talks for schools (2018–2022).
%    \begin{itemize}
%      \item Co‐speaker on GMO \& gene editing seminar for UU alumni (2018)
%      \item Weekend of Science “Plants under the microscope” demo (2019; Dutch)
%      \item Elementary school talks on “Plants of the Future” (2019–2021; Dutch)
%    \end{itemize}
\end{description}


% Selected Projects
%\GreenHeading{Selected Projects}
%\begin{description}
%  \item \textbf{HPC Playground (2025)} – Designed a containerized HPC environment with SLURM and nextflow for young researchers to practice.
  %\item \textbf{Intro to Bioinformatics Course (2022)} – Unified experts from Utrecht Science Park into a cohesive Masters course.
  %\item \textbf{Open Phylogeny Workflow (2021)} – Developed a reproducible JupyterHub workflow for teaching phylogenetics. \myhref{https://github.com/lauralwd/lauras_phylogeny_wf}{github.com/lauralwd/lauras\_phylogeny\_wf}
%\end{description}

% Interests & Volunteering
\GreenHeading{Interests \& Volunteering}
%\begin{multicols}{2}
\begin{description}
  \raggedright
  \item \textbf{Volunteer Facilitator} of monthly discussion groups and retreats for youth on gender fostering inclusive dialogue
  \item \textbf{Sports:} Rock climbing, cycling \& sailing
  \item \textbf{Event Organizer:} “Bèta-dag” for 150+ high-school students and open science workshops
  \item Advanced nature \textbf{photographer:} cover image featured on \emph{Nature Plants}
  %\item Amateur woodworker \& DIY enthusiast—design and build custom lab and home fixtures
  %\item Avid \textbf{gardener} and indoor plant owner
\end{description}
%\end{multicols}

% Skills\& Languages
\noindent
\begin{minipage}[t]{0.6\textwidth}
  \GreenHeading{Technical Proficiencies}
  \begin{multicols}{2}
    \begin{tabular}{p{1em}p{4em}r}
      \textcolor{ForestGreen}{\faRProject}   & R        & \SkillBull{$\bullet\bullet\bullet\bullet\bullet$} \\
      \textcolor{ForestGreen}{\faPython}     & Python   & \SkillBull{$\bullet\bullet\bullet\bullet\circ$} \\
      \textcolor{ForestGreen}{\faTerminal}   & Bash     & \SkillBull{$\bullet\bullet\bullet\bullet\bullet$} \\
      \textcolor{ForestGreen}{\faServer}     & HPC      & \SkillBull{$\bullet\bullet\bullet\bullet\bullet$} \\
      \textcolor{ForestGreen}{\faDocker}     & Docker   & \SkillBull{$\bullet\bullet\bullet\circ\circ$} \\
      \textcolor{ForestGreen}{\faGithub}     & Git      & \SkillBull{$\bullet\bullet\bullet\bullet\circ$} \\
    \end{tabular}
    
    \vfill\null \columnbreak
    
    \begin{tabular}{p{1em}p{8em}r}
      \textcolor{ForestGreen}{\faStream}     & Snakemake           & \SkillBull{$\bullet\bullet\bullet\bullet\bullet$} \\
      \textcolor{ForestGreen}{\faCube}.      & Conda               & \SkillBull{$\bullet\bullet\bullet\bullet\bullet$} \\
      \textcolor{ForestGreen}{\faDatabase}   & SQL                 & \SkillBull{$\bullet\bullet\circ\circ\circ$} \\
      \textcolor{ForestGreen}{\faRobot}      & Machine learning    & \SkillBull{$\bullet\bullet\bullet\bullet\circ$} \\
      \textcolor{ForestGreen}{\faChartBar}   & Visualization       & \SkillBull{$\bullet\bullet\bullet\bullet\bullet$} \\
      \textcolor{ForestGreen}{\faBook}       & Code Notebooks      & \SkillBull{$\bullet\bullet\bullet\bullet\bullet$} \\
    \end{tabular}
  \end{multicols}
\end{minipage}
\hfill
\noindent
\begin{minipage}[t]{.3\textwidth}
\GreenHeading{Languages}
\begin{tabular}{p{1em}p{4em}r}
  \textcolor{ForestGreen}{\faLanguage} & English & \SkillBull{$\bullet\bullet\bullet\bullet\bullet$} \\
  \textcolor{ForestGreen}{\faLanguage} & Dutch   & \SkillBull{$\bullet\bullet\bullet\bullet\bullet$} \\
\end{tabular}
\end{minipage}


\newpage

\GreenHeading{Scientific publications \& PhD chapters}

All publications are listed at \href{https://lauradijkhuizen.com/science}{lauradijkhuizen.com/science}  and 
\textcolor[HTML]{A6CE39}{\faOrcid}\href{https://orcid.org/0000-0002-4628-7671}{ORCID: 0000-0002-4628-7671}

\begin{itemize}
  \raggedright
  \item \textbf{Güngör, E.; Savary, J.; Adema, K.; Dijkhuizen, L.W.; et al.} (2024-07)  
    “The crane fly glycosylated triketide $\delta$‐lactone cornicinine elicits akinete differentiation…,”  
    \emph{Plant, Cell \& Environment}. DOI: \href{https://doi.org/10.1111/pce.14907}{10.1111/pce.14907}

  \item \textbf{Arévalo, S.; Pérez Rico, D.; Abarca, D.; Dijkhuizen, L.W.; et al.} (2024-03-15)  
    “Genome Engineering by RNA-Guided Transposition for \textit{Anabaena} sp.\ PCC 7120,”  
    \emph{ACS Synthetic Biology}. DOI: \href{https://doi.org/10.1021/acssynbio.3c00583}{10.1021/acssynbio.3c00583}

  \item \textbf{Arévalo, S.; Pérez Rico, D.; Abarca, D.; Dijkhuizen, L.W.; et al.} (2022-09-19)  
    “Genome engineering by RNA-guided transposition for \textit{Anabaena} PCC 7120,” Preprint.  
    DOI: \href{https://doi.org/10.1101/2022.09.18.508393}{10.1101/2022.09.18.508393}

  \item \textbf{Güngör, E.; Brouwer, P.; Dijkhuizen, L.W.; et al.} (2021-01)  
    “Azolla ferns testify: seed plants and ferns share a common ancestor…,”  
    \emph{New Phytologist}. DOI: \href{https://doi.org/10.1111/nph.16896}{10.1111/nph.16896}

  \item \textbf{Dijkhuizen, L.W.; Güngör, E.; et al.} (2020-07-24)  
    “LAR phylogeny for Gungor et al. 2020: The complete analysis and dataset,” Dataset.  
    DOI: \href{https://doi.org/10.5281/zenodo.3959057}{10.5281/zenodo.3959057}

\end{itemize}

% PhD Thesis Chapters
\noindent
PhD Thesis chapters

\begin{itemize}
  \setlength{\itemsep}{0.3em}
  \item \textbf{Chapter 1: A hitch-hiker’s guide to Azolla symbiosis genomics} \\
    A broad, less formal introduction to Azolla symbiosis genomics, aimed at engaging a wider scientific audience and providing context for the thesis. \\
    \emph{Laura W. Dijkhuizen}
    % No DOI for this chapter

  \item \textbf{Chapter 2: Foul play in the leaf pocket? The metagenome of floating fern Azolla reveals endophytes that do not fix N\textsubscript{2} but may denitrify} \\
    Discovery and analysis of prokaryotic DNA in Azolla, identification of associated bacterial genomes, and investigation of their metabolic pathways and ecological roles. \\
    \emph{Laura W. Dijkhuizen, et al.}
    \href{https://doi.org/10.1111/nph.14843}{DOI}

  \item \textbf{Chapter 3: Hidden treasures: public sequencing data of symbiotic Azolla ferns harbours a genus-wide metagenome} \\
    Development of a workflow to enrich and study genomes of bacteria associated with all sequenced Azolla species, revealing systematic presence and vertical transfer of key symbionts. \\
    \emph{Laura W. Dijkhuizen, et al.}
    % No DOI for this chapter

  \item \textbf{Appendix B: Metagenomics practical} \\
    An educational practical designed to teach metagenomics principles and techniques to Life Sciences students, using Bash and Jupyter notebooks for hands-on learning. \\
    \emph{Laura W. Dijkhuizen}
    \href{https://github.com/lauralwd/metagenomicspractical}{GitHub}

  \item \textbf{Chapter 4: Forever together: One Nostoc azollae is symbiont to all Azolla species} \\
    Comparative genomics of the main Azolla symbiont, N. azollae, showing near-identical genomes across hosts, high pseudogene content, and phylogenomic placement within Nostocales. \\
    \emph{Laura W. Dijkhuizen, et al.}
    % No DOI for this chapter

  \item \textbf{Chapter 5: It takes two: Far-Red light induces the Azolla-Nostoc symbiosis sexual reproduction} \\
    Investigation of sexual reproduction and symbiont transmission in Azolla/N. azollae, including environmental triggers, gene regulation, and evolutionary implications for crop application. \\
    \emph{Laura W. Dijkhuizen, Tabatabaei, B.E.S., Brouwer, P., et al.}
    \href{https://doi.org/10.1101/2020.09.09.289736}{DOI}

  \item \textbf{Chapter 6: One, Two, Tree! A workflow for creating state-of-the-art phylogenies designed for reproducibility with JuPyter, conda and git} \\
    Description of a reproducible workflow for phylogenetic tree inference in land plants, using open-source tools and providing resources for semi-automatic tree annotation. \\
    \emph{Laura W. Dijkhuizen}
    % No DOI for this chapter
\end{itemize}

\end{document}
% cv.tex
%—————————————————————————————————––
% Simple asymmetric two-column CV using vars.tex for overrides
%—————————————————————————————————––
\documentclass[a4paper,10pt]{article}
\usepackage[vmargin=1.5cm, hmargin=1.5cm]{geometry}

% Load general setup (fonts, spacing, tikz, adjustbox, etc.)
\input{MySetup}

% Load branch-specific variables
% vars.tex
% ———————––
% Overrideable variables for CV

%––––––––––––––––––––
% Personal details
\newcommand{\FullName}{Laura W. Dijkhuizen}
%\newcommand{\BirthDate}{}
%\newcommand{\BirthPlace}{}
\newcommand{\Residence}{Utrecht, the Netherlands}

\newcommand{\EmailAddress}{laura.w.dijkhuizen@pm.me}
\newcommand{\WebPage}{https://lauradijkhuizen.com}

%––––––––––––––––––––
% File paths
\newcommand{\ProfilePhoto}{pf.jpg}
%\newcommand{\BackgroundImage}{bg-image.pdf}% e.g. a faint watermark or logo

%––––––––––––––––––––
% Style colors (hex RGB without #)
\definecolor{ForestGreen}{HTML}{14532D}


% Go Dutch
\usepackage[dutch]{babel}

% Define a custom gray color in your preamble (or near the top of your document)
\definecolor{BracketGray}{HTML}{888888}

%—————————————————————————————————––
\begin{document}
\pagestyle{fancy}
\thispagestyle{empty}

% setup column nr 1 —————————————————————————————————––
\noindent\makebox[\textwidth][s]{
\begin{adjustbox}{valign=t}
\noindent\begin{minipage}[t]{0.3\textwidth}

\begin{center}
  \begin{tikzpicture}
      \clip (0,0) ellipse (2cm and 2.5cm);
      \node {\includegraphics[width=4cm]{\ProfilePhoto}};
  \end{tikzpicture}

{\begin{spacing}{.8}
  \LARGE \headerfont \FullName
\end{spacing}}

\MySkip

  \ifdefined\BirthDate Born on \BirthDate\\ \fi
  \ifdefined\BirthPlace \BirthPlace\\ \fi
  \ifdefined\Residence \Residence\\ \fi

\MySkip
\textcolor{ForestGreen}{\faEnvelope} \href{mailto:\EmailAddress}{\EmailAddress}\\
\textcolor{ForestGreen}{\faGlobe} \href{\WebPage}{\WebPage}
\end{center}

% About me — R&D data & reporting (tailored)

{\RaggedRight

  Mensgericht \textbf{verbinder} op het snijvlak van \textbf{data}, \textbf{governance} en \textbf{kwaliteitsbewaking}. 
  Combineert hands-on ervaring met het opzetten van datacatalogi, metadata‑systemen en FAIR infrastructuur met verbindend leiderschap, stakeholdercoördinatie en strategisch inzicht.

}

\GreenHeading{Kwaliteiten}
\begin{itemize}[leftmargin=1em, itemindent=0em]
  \item \textbf{Mensgerichte coördinator:}\\ 
    Leidde team van 20 PhD’s; vertegenwoordigde 2.000+ promovendi in beleidsoverleg. Ervaren aanspreekpunt voor data- en analysevragen.
  \item \textbf{Governance en kwaliteitsborging:}\\ 
    Ontwierp datacatalogus en metadata‑systemen; borgde standaarden en traceerbaarheid.
  \item \textbf{Reproduceerbare workflows:} \\ 
    Bouwde veilige FAIR-infrastructuur voor opslag, analyse, Snakemake workflows en documentatie.
  \item \textbf{Interdisciplinaire verbinder:} \\ 
    Brug tussen IT, beleid en inhoud; gewend aan afstemming tussen disciplines.
  \item \textbf{Strategisch en procesgericht:} \\ 
    Coördineerde datatrainingen voor promovendi en professionals (500+ deelnemers).
  \item \textbf{Vertrouwd beleidsadviseur:} \\ 
    Adviseur bij TU Delft en UU; actief in strategische commissies rond data, onderwijs en welzijn.
\end{itemize}

% Close column 1 put in vertical line as column2 —————————————————————————————————––

\end{minipage}%
\end{adjustbox}%
\hfill%


\begin{adjustbox}{valign=t}
\hfill%
\begin{minipage}[t]{0.05\textwidth}
\MyVerticalRule
\end{minipage}%
\end{adjustbox}
% Close column 2 and setup column nr 3 —————————————————————————————————––

\begin{adjustbox}{valign=t}
\hfill%
\begin{minipage}[t]{0.6\textwidth}

% Contents column 3 —————————————————————————————————––

% Employment
\GreenHeading{Werkervaring \& Opleiding}
\begin{description}
  \item[\textcolor{ForestGreen}{\textbf{2022 -- heden}}] 
  \textbf{Coördinator datatraining \& trainer}  
  Verantwoordelijk voor ontwerp, uitvoering en coördinatie van trainingen in data-analyse, datavisualisatie en governance voor professionals. 
  Coach voor promovendi; borging van FAIR-principes en reproduceerbaarheid.  
  (\href{https://github.com/lauralwd/professional_education}{GitHub})\\
  \textit{Universiteit Utrecht}

  \item[\textcolor{ForestGreen}{\textbf{2017 -- 2022}}] 
  \textbf{Projectleider data-infrastructuur (PhD)}  
  Richtte veilige data-infrastructuur op; ontwikkelde datacatalogus en reproduceerbare workflows. Adviseur voor kwaliteitsbewaking en analysebesluiten.  
  (\href{https://github.com/lauralwd/azolla_phd_thesis}{GitHub})\\
  \textit{Universiteit Utrecht}

  \item[\textcolor{ForestGreen}{\textbf{2010 -- 2017}}] 
  \textbf{MSc / BSc Environmental Plant Biology}  
  Data-intensieve opleiding met focus op biologie, modellering en duurzaamheid.  
  \textit{Universiteit Utrecht}
\end{description}

% Management
\GreenHeading{Teamleiding, Beleid \& Advies}
\begin{description}

  \item[\textcolor{ForestGreen}{\textbf{2017 -- 2021}}] 
  \textbf{Voorzitter vertegenwoordiging promovendi UU}  
  Vertegenwoordigde 2.000+ promovendi richting decanen, HR en strategisch beleid.  
  Leidde een team van 20 vertegenwoordigers en coördineerde belangenafstemming over faculteiten heen.  
  Initiatiefnemer van nieuwe monitoringsstructuur met indicatoren voor beleidsevaluatie.

  \item[\textcolor{ForestGreen}{\textbf{2019 -- heden}}] 
  Lid van diverse \textbf{strategische commissies} binnen de Universiteit Utrecht.  
  Brug tussen uitvoering en beleid (Open Science Platform, implementatieteam, MSc Bioinformatics).

  \item[\textcolor{ForestGreen}{\textbf{2021 -- 2022}}] 
  \textbf{Externe beleidsadviseur TU Delft}  
  Lid van onafhankelijke visitatiecommissie voor evaluatie van strategie, prestaties en borging van kwaliteitsbeleid.  
  Adviezen opgenomen in eindrapport senior management.  
  (\href{https://filelist.tudelft.nl/TUDelft/Onderzoek/Kwaliteitsborging/Final report SEP Chemistry TU Delft 20220204.pdf}{Rapport})
\end{description}


% Key Achievements & Metrics
\GreenHeading{Projectresultaten en Impact}
\begin{description}
  \item Ontwikkelde en coördineerde trainingen in R, Python, datavisualisatie en data governance voor promovendi en professionals.  
  Aangepast op diverse achtergronden, waaronder zorg, labonderzoek en milieu (500+ deelnemers, gemiddeld 8+/10).  
  Ervaren in schakelen tussen domeinen en leerdoelen.

  \item Richtte een labbrede data-infrastructuur op en beheerde deze.  
  Bouwde datacatalogus, versiebeheer, toegangsstructuur en workflows waarmee 30+ gebruikers reproduceerbaar en gedeeld konden werken.

  \item Initieerde en leidde interfacultaire werkgroep rond voortgang en welzijn van promovendi.  
  Governancekader bereikt 3.500+ promovendi.

  \item Ondersteunde professionals bij het prioriteren en oplossen van data-issues.  
  Rol als aanspreekpunt bij interpretatieproblemen, inconsistenties en datakwaliteit in diverse settings.

  \item Vertaalde complexe analyse‑uitvoer naar dashboards en heldere inzichten voor collega’s en belanghebbenden.

  \item Ontwikkelde data workflows voor analyse en dashboarding van biologische data.  
  (\href{https://github.com/lauralwd/anabaena_nanopore_workflow}{variant calling}, 
  \href{https://github.com/lauralwd/Nostoc_azollae_pangenomics}{pangenomics},
  \href{https://github.com/lauralwd/lauras_phylogeny_wf}{phylogeny},
  \href{https://github.com/lauralwd/Azolla_genus_meta_pangenome}{metagenome})
\end{description}

% Contents page 2 —————————————————————————————————––
\end{minipage}%
\end{adjustbox}%
}

\newpage

% Training & Knowledge sharing
\GreenHeading{Training \& Kennisdeling}
\begin{description}
  \item[\textcolor{ForestGreen}{\textbf{2022 -- heden}}] 
  \textbf{Trainer \& coördinator datavaardigheden}  
  Ontwikkeling en uitvoering van trainingen in R, Python, datavisualisatie, workflows en data governance.  
  Aandacht voor reproduceerbaarheid en het effectief communiceren van data-inzichten.
  
  \item[\textcolor{ForestGreen}{\textbf{2020 -- 2021}}] 
  \textbf{Organisator mastercursus data-integratie}  
  Coördineerde experts uit verschillende vakgebieden in één hands-on cursus over bioinformatica.  
  (\href{https://lauralwd.github.io/metagenomicspractical/}{GitHub})

  \item[\textcolor{ForestGreen}{\textbf{2017 -- 2022}}] 
  \textbf{Begeleiding van studenten}  
  Coachte studenten bij data-analyse, visualisatie en reproduceerbare werkwijzen.
\end{description}

\vfill

% Communication & Stakeholder engagement
\GreenHeading{Communicatie \& Stakeholderengagement}
\begin{description}
  \item Regelmatige deelnemer aan landelijke bijeenkomsten over open data, datastandaarden en beleidsvorming (o.a. VSNU, data.overheid.nl).

  \item Actieve spreker en organisator bij sessies over reproduceerbaarheid, open science en datagovernance.

  \item Publiekslezingen en demonstraties op scholen en open dagen over genetica en data-inzichten.
\end{description}

\GreenHeading{Procesverbetering \& Datakwaliteit}
\begin{description}
  \item Drijvende kracht achter campus-breed dataeducatie programma voor professionals. Inclusief roadmap "iedereen AI-ready in vijf jaar".
  \item Ontwikkelde afspraken voor bestandsnamen, folderstructuren en versiebeheer voor analyseprojecten.  
  Zorgde daarmee voor eenduidigheid, terugvindbaarheid en gedeeld gebruik.
  \item Continue agile-like verbeteringen van trainingsmateriaal op basis van deelnemer-feedback
  \item Voerde kwaliteitschecks uit op data en metadata binnen projecten; gaf gerichte feedback aan gebruikers.
\end{description}

% Scientific Publications selection
\GreenHeading{Selectie van Publicaties}
\begin{description}
  \item \textbf{Reproduceerbare data-analyse:} Publiceerde volledige dataset en workflow voor fylogenie van varens.  
  \href{https://doi.org/10.1111/nph.16896}{\emph{New Phytologist} (2021)}

  \item \textbf{Complexe data-inzichten:} Onderzoek naar stikstoffixatie en denitrificatie in symbiotische systemen.  
  \href{https://doi.org/10.1111/nph.14843}{\emph{New Phytologist} (2018)}
\end{description}

\vfill

% Interests & Volunteering
\GreenHeading{Interesses \& Vrijwilligerswerk}
\begin{description}
  \raggedright
  \item \textbf{Vrijwilliger en facilitator} van maandelijkse discussiegroepen en begeleider tijdens meerdaagse kampen voor jongeren over gender en inclusie.
  % \item \textbf{Wetenschapscommunicatie \& outreach}: Ervaren spreker op conferenties en publieksbijeenkomsten. Medeorganisator van open science workshops en spreker op onderwijsinitiatieven binnen de universiteit.
  \item \textbf{Actief buitenmens}: Klimmen, fietsen en zeilen — vaak ook met collega’s.
  % \item \textbf{Bèta-dag coördinator}: Organiseerde evenementen voor 150+ middelbare scholieren om kennis te maken met wetenschap en onderzoek.
  \item \textbf{Fotografie}: Gevorderd natuurfotograaf met publicatie op de cover van \href{https://lauralwd.github.io/photography/}{\emph{Nature Plants} (link)}.
  \item \textbf{Groene vingers}: Enthousiast tuinierder en plantenliefhebber.
  \item \textbf{maker} Raspberry PI, sensor netwerken, 3d printers en automatiseering.
\end{description}
\null

\vfill

% Skills\& Languages
\noindent
\begin{minipage}[t]{0.6\textwidth}
    \GreenHeading{Technische en data vaardigheden}
    \begin{tabular}{p{1em}p{11em}p{14.5em}r}
      \textcolor{ForestGreen}{\faServer}   & HPC \& Workflow Ontwerp  & \textcolor{BracketGray}{(Conda, Snakemake)}                   & \SkillBull{$\bullet\bullet\bullet\bullet\bullet$} \\
      \textcolor{ForestGreen}{\faDatabase} & SQL / Data Governance   & \textcolor{BracketGray}{(SQL, iRODS, FAIR)}                    & \SkillBull{$\bullet\bullet\bullet\circ\circ$} \\
      \textcolor{ForestGreen}{\faRProject} & R / Python / Bash       & \textcolor{BracketGray}{(Geavanceerde scripts, Sysadmin)}       & \SkillBull{$\bullet\bullet\bullet\bullet\bullet$} \\
      \textcolor{ForestGreen}{\faDocker}   & Docker / GitHub         & \textcolor{BracketGray}{(Actions, CI/CD)}                     & \SkillBull{$\bullet\bullet\bullet\bullet\circ$} \\
      \textcolor{ForestGreen}{\faChartBar} & Data Dashboards         & \textcolor{BracketGray}{(ggplot2, Shiny, Plotly)}              & \SkillBull{$\bullet\bullet\bullet\bullet\bullet$} \\
      \textcolor{ForestGreen}{\faBook}     & Reproducible Notebooks  & \textcolor{BracketGray}{(Jupyter, Quarto)}                     & \SkillBull{$\bullet\bullet\bullet\bullet\bullet$} \\
    \end{tabular}
\end{minipage}
\hfill
\noindent
\begin{minipage}[t]{.3\textwidth}
\GreenHeading{Talen}
\begin{tabular}{p{1em}p{4em}r}
  \textcolor{ForestGreen}{\faLanguage} & Engels & \SkillBull{$\bullet\bullet\bullet\bullet\bullet$} \\
  \textcolor{ForestGreen}{\faLanguage} & Nederlands   & \SkillBull{$\bullet\bullet\bullet\bullet\bullet$} \\
\end{tabular}
\end{minipage}

\end{document}
% cv.tex
%—————————————————————————————————––
% Simple asymmetric two-column CV using vars.tex for overrides
%—————————————————————————————————––
\documentclass[a4paper,10pt]{article}
\usepackage[vmargin=1.5cm, hmargin=1.5cm]{geometry}

% Load general setup (fonts, spacing, tikz, adjustbox, etc.)
\input{MySetup}

% Load branch-specific variables
% vars.tex
% ———————––
% Overrideable variables for CV

%––––––––––––––––––––
% Personal details
\newcommand{\FullName}{Laura W. Dijkhuizen}
%\newcommand{\BirthDate}{}
%\newcommand{\BirthPlace}{}
\newcommand{\Residence}{Utrecht, the Netherlands}

\newcommand{\EmailAddress}{laura.w.dijkhuizen@pm.me}
\newcommand{\WebPage}{https://lauradijkhuizen.com}

%––––––––––––––––––––
% File paths
\newcommand{\ProfilePhoto}{pf.jpg}
\newcommand{\BackgroundImage}{bg-image.pdf}% e.g. a faint watermark or logo

%––––––––––––––––––––
% Style colors (hex RGB without #)
\definecolor{ColorOne}{HTML}{003399}   % Primary accent color
\definecolor{ColorTwo}{HTML}{FF6600}   % Secondary accent color

%––––––––––––––––––––
% Icon definitions (if needed)
% \newcommand{\IconEmail}{\faEnvelopeO}
% \newcommand{\IconLink}{\faChain}

%––––––––––––––––––––
% Any other branch-specific overrides go here

%—————————————————————————————————––
\begin{document}
\pagestyle{fancy}
\thispagestyle{empty}

% setup column nr 1 —————————————————————————————————––
\noindent\makebox[\textwidth][s]{
\begin{adjustbox}{valign=t}
\noindent\begin{minipage}[t]{0.3\textwidth}

\begin{center}
  \begin{tikzpicture}
      \clip (0,0) ellipse (2cm and 2.5cm);
      \node {\includegraphics[width=4cm]{\ProfilePhoto}};
  \end{tikzpicture}

{\begin{spacing}{.8}
  \LARGE \bfseries \FullName
\end{spacing}}

\MySkip

\fontfamily{pag}\selectfont
  \ifdefined\BirthDate Born on \BirthDate\\ \fi
  \ifdefined\BirthPlace \BirthPlace\\ \fi
  \ifdefined\Residence Currently living in \Residence\\ \fi

\MySkip
\fontfamily{pag}\selectfont
\textcolor{ForestGreen}{\faEnvelope} \href{mailto:\EmailAddress}{\EmailAddress}\\
\textcolor{ForestGreen}{\faGlobe} \href{\WebPage}{\WebPage}
\end{center}

% About me — R&D data & reporting (tailored)
{\fontfamily{pag}\selectfont\RaggedRight
  Passionate about \textbf{data \& reporting for plant science} — turning complex data into actionable insights. \\
  I thrive on \textbf{leading collaborative projects}, 
  and communicating across technical and non-technical teams.
}

% Professional Attributes
{\fontfamily{pag}\selectfont\RaggedRight

\GreenHeading{Professional Attributes}
\begin{spacing}{1}
\begin{itemize}
  \setlength{\itemindent}{-1em}
  \setlength{\itemsep}{0.0em}
  \item Proven track record in cross-faculty governance and stakeholder leadership, representing over 2000 PhDs.
  \item Highest-rated \textbf{educator} (2 yrs) for PhD and postdoc courses
  % \item Led cross-faculty \textbf{governance initiative} impacting 2000+ PhDs; policy adopted
  % \item Experienced \textbf{chair} \& consensus builder representing $\sim$2000 PhDs
  \item \textbf{Autonomous multitasker} with project leadership
  \item Clear \textbf{communicator} across technical \& non-technical audiences
  \item \textbf{Genomics data scientist} (R, Python, Bash/Linux) with strong track record in R\&D data governance.
  \item \textbf{HPC \& workflow specialist}: Docker, GitHub, reproducible pipelines, code notebooks and data dashboards
  \item \textbf{Domain specialist} on plant symbiosis biology, genomics \& environmental research
\end{itemize}
\end{spacing}
}


% Data governance is	
%   •	Accurate (quality, validation, error handling)
% 	•	Consistent (standard formats, master data management)
% 	•	Secure (access controls, compliance, privacy)
% 	•	Accessible & usable (so the right people get the right data in the right form at the right time)
% 	•	Accountable (clear ownership and stewardship of datasets)

% Close column 1 put in vertical line as column2 —————————————————————————————————––

\end{minipage}%
\end{adjustbox}%
\hfill%


\begin{adjustbox}{valign=t}
\hfill%
\begin{minipage}[t]{0.05\textwidth}
\MyVerticalRule
\end{minipage}%
\end{adjustbox}
% Close column 2 and setup column nr 3 —————————————————————————————————––

\begin{adjustbox}{valign=t}
\hfill%
\begin{minipage}[t]{0.6\textwidth}

% Contents column 3 —————————————————————————————————––

% Employment
\GreenHeading{Employment \& Education}
\begin{description}
\raggedright
  \item[\normalfont \textcolor{ForestGreen}{\textbf{2022 -- now.}}] \textbf{Lecturer \&  Trainer} 
    in programming, bioinformatics \& data science for PhD candidates \& postdocs 
    (\href{https://github.com/lauralwd/professional_education}{GitHub})\\
    \textit{Theoretical Biology \& Bioinformatics Group at Utrecht University}
  \item[\normalfont \textcolor{ForestGreen}{\textbf{2017 -- 2022.}}] \textbf{PhD Researcher and teacher} 
    I secured €250k funding for my own PhD on the (meta)genomics of novel crop \textit{Azolla} 
    (\href{https://github.com/lauralwd/azolla_phd_thesis}{GitHub})\\ 
    \textit{Molecular Plant Physiology Group, Utrecht University.}
  \item[\normalfont \textcolor{ForestGreen}{\textbf{2010 -- 2017.}}] \textbf{MSc / BSc} Environmental Plant Biology \\
    \textit{Utrecht University}
\end{description}

% Management
\GreenHeading{Leadership \& People Management}

% \emph{Coordinated 8+ expert instructors; chaired council representing >2,000 PhDs; frequent liaison with deans/HR/graduate schools to drive policy and adoption.}
\begin{description}
\raggedright
\item[\normalfont \textcolor{ForestGreen}{\textbf{2021 -- 2022.}}] Appointed to \textbf{independent expert} panel reviewing R\&D strategy, performance, and governance 
  for TU Delft’s Chemical Engineering \& Biotechnology departments; findings informed senior leadership decisions 
  (\href{https://filelist.tudelft.nl/TUDelft/Onderzoek/Kwaliteitsborging/Final report SEP Chemistry TU Delft 20220204.pdf}{Report}).
\item[\normalfont \textcolor{ForestGreen}{\textbf{2018.}}] \textbf{Interim-lead} of lab during leadership absence.
\item[\normalfont \textcolor{ForestGreen}{\textbf{2019 -- now.}}] Member of multiple strategic committees at Utrecht University, 
contributing to governance, policy development, reproducibility initiatives, and curriculum design:
  \begin{itemize}
    \item \textbf{Department Advisory Committee} (Biology) %– feedback on strategic priorities and governance.
    \item \textbf{Faculty Open Science Implementation Team} \& \textbf{UU Open Science Platform} %– driving reproducibility policies and innovations 
      (\href{https://www.uu.nl/en/news/meet-laura-dijkhuizen}{Interview}).
    \item \textbf{Curriculum Committee} for the M.Sc.\ Bioinformatics \& Biocomplexity %– designed a brand new Master’s programme 
    (\href{https://www.uu.nl/en/masters/bioinformatics-and-biocomplexity}{link}).
  \end{itemize}
  % \item[\normalfont \textcolor{ForestGreen}{\textbf{2023 -- now.}}] Member, \textbf{Department Advisory Committee}
  %   , Biology Dept., Utrecht University.
  %   -- feedback on strategic priorities \& governance.
  % \item[\normalfont \textcolor{ForestGreen}{\textbf{2020 -- 2021.}}] Member, \textbf{Faculty Open Science Implementation Team} 
  %   \& \textbf{UU Open Science Platform}. -- driving reproducibility policies and innovations 
  %   (\href{https://www.uu.nl/en/news/meet-laura-dijkhuizen}{Interview}).
  % \item[\normalfont \textcolor{ForestGreen}{\textbf{2019 -- 2021.}}] Member, \textbf{Curriculum Committee}
  %   , M.Sc. Bioinformatics\& Biocomplexity. -- Design a brand new Masters. 
  %   (\href{https://www.uu.nl/en/masters/bioinformatics-and-biocomplexity}{link})
  \item[\normalfont \textcolor{ForestGreen}{\textbf{2017 -- 2021.}}] \textbf{PhD Council Chair} 
    and member \textbf{Board of Studies} (GS-LS) 
      -- led council representing ~2,000 PhDs; coordinated priorities, delegated and monitor tasks, 
      and built consensus among stakeholders.”
    % \item[\normalfont \textcolor{ForestGreen}{\textbf{2017 -- 2021.}}] \textbf{PhD Council Representative}
  %   , Institute of Environmental Biology.
\end{description}


% Key Achievements & Metrics
\GreenHeading{Key Achievements}
\begin{description}
%\begin{itemize}
%    \setlength{\itemindent}{0em}
%    \setlength{\parindent}{5em}
  \item Trained 500+ professionals in my self made bioinformatics curriculum for professionals. 
    (\href{https://github.com/lauralwd/professional_education}{GitHub})
  Designed, built, and delivered a brand-new bioinformatics course within weeks.%; iterated on MVP after each run based on learner feedback.”
  %\item Taught “Intro to Bioinformatics” course; cohesively uniting various domain experts from Utrecht Science Park.
  % \item Designed and deployed reproducible pipelines, shareable analysis notebooks and data dashboards advancing R\&D  and data governance. 
    % Pipelines include  
    %   Nanopore variant calling (\href{https://github.com/lauralwd/anabaena_nanopore_workflow}{link})
    %   , pangenomics (\href{https://github.com/lauralwd/Nostoc_azollae_pangenomics}{link})
    %   , phylogeny (\href{https://github.com/lauralwd/lauras_phylogeny_wf}{link})
    %   , and metagenome analysis (\href{https://github.com/lauralwd/Azolla_genus_meta_pangenome}{link}). 
  % \item Chaired a council representing $\sim$2000 PhD candidates,
  %   initiated cross-faculty working group with deans, HR, and graduate schools to establish shared progress, well-being and graduation metrics, 
  %   adopted as official governance framework.
  % \item Experienced stakeholder manager and communicator serving on 5 more strategic committees
  %   (See Management section).
  \item Chaired council representing $\sim$2000 PhD candidates across three faculties 
  and initiated cross-faculty working group with deans, HR, and graduate schools 
  to establish shared progress and well-being metrics, adopted as official governance framework 
  — improving transparency, data quality, and decision-making for all doctoral trajectories.
  %\item Interim led a small lab during long-term leadership absence.
  \item Experienced public speaker from elementary schools to scientific conferences and my courses for research professionals
    (\href{https://lauralwd.github.io/outreach/}{Outreach page}).
  %\item Featured in many media outlets (national newspaper, radio \& local TV), bringing complex science to the public.
  \item Build and maintained a bioinformatics server 
    (\href{https://lauralwd.github.io/blog/post-mpp-server/}{link}) 
    enabling experimentalist colleagues to use my infrastructure with ease such as interactive code notebooks and dashboards.
  \item Worked with a wide variety of data. 
    From PacBio long reads to microRNA-seq and from optical mapping to handwritten notes.
    I often focus on data analysis in tight collaboration with experimentalists.
%\end{itemize}
\end{description}

% Contents page 2 —————————————————————————————————––
\end{minipage}%
\end{adjustbox}%
}
\newpage

% Teaching & Training
\GreenHeading{Teaching \& Training}
\begin{description}
  \raggedright
  \item \textbf{Lecturer \& Trainer (2022–now)}: Design and deliver courses in 
    basic and advanced R, Python \& Bash skills for PhD candidates and postdocs, 
%    tailoring content to individual research goals and team projects 
    including data governance tools for stakeholder communication.

    (\href{https://github.com/lauralwd/professional_education}{GitHub}).
  \item \textbf{Course Coordinator (2020–2021)}: Led the “Introduction to Bioinformatics” Master’s course 
    -- integrating 8+ domain experts 
    from across Utrecht Science Park into a unified, hands-on curriculum 
    (\href{https://lauralwd.github.io/metagenomicspractical/}{GitHub}).
  \item \textbf{Thesis Supervisor (2017–2022)}: Mentored 10+ MSc/BSc students on computational biology and 
    plant physiology projects guiding experimental design, data pipelines, and results communication.
  \item \textbf{Workshop Facilitator \& TA (2017–2020)}: Supported courses in molecular biology, genetics, and bioinformatics
  %\item \textbf{Educational Innovator}: Developed Docker-based teaching environments and collaborative online coding exercises.
  % HPC playground
\end{description}


% % Outreach & Communication
% \GreenHeading{Outreach \& Communication}
% \begin{description}
%   \raggedright
%   \item \textbf{Television:} Local TV interview on \textit{Azolla} ferns (2017; Dutch) 
%     -- \href{https://youtu.be/OI4VV4M2-f4}{Watch}. 
%     And an interview for “De Kennis van Nu” popular science program (2018; Dutch) 
%     -- \href{https://ntr.nl/Focus/287/detail/Onkruid-als-reddende-engel/VPWON_1292624}{Watch}.
%   \item \textbf{Radio:} BNR national radio feature on Azolla (2017; Dutch) 
%     -- \href{https://www.bnr.nl/podcast/wetenschap-vandaag/10346708/utrechts-plantje-geniet-wereldwijde-faam}{Link}
%   \item \textbf{Print:} Feature in AD newspaper (2018; Dutch) 
%     -- \href{https://www.ad.nl/utrecht/kroosachtig-plantje-uit-sloot-naast-galgenwaard-blijkt-ware-eiwitbom~a1eaba6d/}{Read}
% %  \item \textbf{Online:} UU News interview on Open Science views (2021; EN/NL)
%     % -- \href{https://www.uu.nl/en/news/meet-laura-dijkhuizen}{Read}
%   \item \textbf{Lectures \& Events:} Frequent invited speaker and demonstrator at public science events,
%     including gene editing seminars, hands-on plant biology demos, 
%     and outreach talks for schools (2018–2022).
% %    \begin{itemize}
% %      \item Co‐speaker on GMO \& gene editing seminar for UU alumni (2018)
% %      \item Weekend of Science “Plants under the microscope” demo (2019; Dutch)
% %      \item Elementary school talks on “Plants of the Future” (2019–2021; Dutch)
% %    \end{itemize}
% \end{description}


% SCientific skills and papers
\GreenHeading{Selected Scientific Publications}

All publications are listed at \href{https://lauradijkhuizen.com/science}{lauradijkhuizen.com/science}  and 
\textcolor[HTML]{A6CE39}{\faOrcid}\href{https://orcid.org/0000-0002-4628-7671}{ORCID: 0000-0002-4628-7671}

\textbf{Genomics \& Bioinformatics}
  \begin{itemize}
    \item \textbf{Genome Engineering by RNA-Guided Transposition for \textit{Anabaena} sp.}  
      ACS Synthetic Biology (2024). \href{https://doi.org/10.1021/acssynbio.3c00583}{DOI}
    \item \textbf{Is there foul play in the leaf pocket?}  
      New Phytologist (2018). \href{https://doi.org/10.1111/nph.14843}{DOI}
      \item \textbf{Azolla ferns testify: seed plants and ferns share a common ancestor for LAR}  
        New Phytologist (2021). \href{https://doi.org/10.1111/nph.16896}{DOI}
  \end{itemize}

\noindent\textbf{Workflow \& Reproducible Research}
  \begin{itemize}
    \item \textbf{LAR phylogeny for Gungor et al. 2020: The complete analysis and dataset}  
      Dataset (2020-07-24). \href{https://doi.org/10.5281/zenodo.3959057}{DOI}
    \item \textbf{Chapter 3: Hidden treasures: public sequencing data of symbiotic Azolla ferns harbours a genus-wide metagenome}
      Thesis chapter phd repo link
  \end{itemize}

\noindent\textbf{Plant Physiology \& Ecology}
  \begin{itemize}
    \item \textbf{The crane fly glycosylated triketide $\delta$-lactone cornicinine elicits akinete differentiation of the cyanobiont in aquatic \textit{Azolla} fern symbioses}  
      Plant, Cell \& Environment (2024). \href{https://doi.org/10.1111/pce.14907}{DOI}
    \item \textbf{Control of the \textit{Azolla} symbiosis sexual reproduction: ferns to shed light on the origin of floral regulation?}  
      Preprint (2020). \href{https://doi.org/10.3389/fpls.2021.693039}{DOI}
  \end{itemize}



% Interests & Volunteering
\GreenHeading{Interests \& Volunteering}
%\begin{multicols}{2}
\begin{description}
  \raggedright
  \item \textbf{Volunteer Facilitator} of monthly discussion groups and camp councillor on bi-yearly retreats for youth on gender fostering inclusive dialogue
  \item \textbf{Sports:} Rock climbing, cycling \& sailing
  % \item \textbf{Event Organizer:} “Bèta-dag” for 150+ high-school students and open science workshops
  \item Advanced nature \textbf{photographer:} cover image featured on \emph{Nature Plants} (\href{https://lauralwd.github.io/photography/}{Link})
  %\item Amateur woodworker \& DIY enthusiast—design and build custom lab and home fixtures
  %\item Avid \textbf{gardener} and indoor plant owner
\end{description}
%\end{multicols}

% Skills\& Languages
\noindent
\begin{minipage}[t]{0.6\textwidth}
  \GreenHeading{Technical \& Data Governance Proficiencies}
  \begin{multicols}{2}
    \begin{tabular}{p{1em}p{4em}r}
      \textcolor{ForestGreen}{\faRProject}   & R        & \SkillBull{$\bullet\bullet\bullet\bullet\bullet$} \\
      \textcolor{ForestGreen}{\faPython}     & Python   & \SkillBull{$\bullet\bullet\bullet\bullet\circ$} \\
      \textcolor{ForestGreen}{\faTerminal}   & Bash     & \SkillBull{$\bullet\bullet\bullet\bullet\bullet$} \\
      \textcolor{ForestGreen}{\faServer}     & HPC      & \SkillBull{$\bullet\bullet\bullet\bullet\bullet$} \\
      \textcolor{ForestGreen}{\faDocker}     & Docker   & \SkillBull{$\bullet\bullet\bullet\circ\circ$} \\
      \textcolor{ForestGreen}{\faGithub}     & Git      & \SkillBull{$\bullet\bullet\bullet\bullet\circ$} \\
    \end{tabular}
    
    \vfill\null \columnbreak
    
    \begin{tabular}{p{1em}p{8em}r}
      \textcolor{ForestGreen}{\faStream}     & Snakemake           & \SkillBull{$\bullet\bullet\bullet\bullet\bullet$} \\
      \textcolor{ForestGreen}{\faCube}.      & Conda               & \SkillBull{$\bullet\bullet\bullet\bullet\bullet$} \\
      \textcolor{ForestGreen}{\faDatabase}   & SQL                 & \SkillBull{$\bullet\bullet\circ\circ\circ$} \\
      \textcolor{ForestGreen}{\faRobot}      & Machine learning    & \SkillBull{$\bullet\bullet\bullet\bullet\circ$} \\
      \textcolor{ForestGreen}{\faChartBar}   & Data dashboards      & \SkillBull{$\bullet\bullet\bullet\bullet\bullet$} \\
      \textcolor{ForestGreen}{\faBook}       & Code Notebooks      & \SkillBull{$\bullet\bullet\bullet\bullet\bullet$} \\
    \end{tabular}
  \end{multicols}
\end{minipage}
\hfill
\noindent
\begin{minipage}[t]{.3\textwidth}
\GreenHeading{Languages}
\begin{tabular}{p{1em}p{4em}r}
  \textcolor{ForestGreen}{\faLanguage} & English & \SkillBull{$\bullet\bullet\bullet\bullet\bullet$} \\
  \textcolor{ForestGreen}{\faLanguage} & Dutch   & \SkillBull{$\bullet\bullet\bullet\bullet\bullet$} \\
\end{tabular}
\end{minipage}


\newpage

\GreenHeading{Scientific publications \& PhD chapters}

All publications are listed at \href{https://lauradijkhuizen.com/science}{lauradijkhuizen.com/science}  and 
\textcolor[HTML]{A6CE39}{\faOrcid}\href{https://orcid.org/0000-0002-4628-7671}{ORCID: 0000-0002-4628-7671}

\begin{itemize}
  \raggedright
  \item \textbf{Güngör, E.; Savary, J.; Adema, K.; Dijkhuizen, L.W.; et al.} (2024-07)  
    “The crane fly glycosylated triketide $\delta$‐lactone cornicinine elicits akinete differentiation…,”  
    \emph{Plant, Cell \& Environment}. DOI: \href{https://doi.org/10.1111/pce.14907}{10.1111/pce.14907}

  \item \textbf{Arévalo, S.; Pérez Rico, D.; Abarca, D.; Dijkhuizen, L.W.; et al.} (2024-03-15)  
    “Genome Engineering by RNA-Guided Transposition for \textit{Anabaena} sp.\ PCC 7120,”  
    \emph{ACS Synthetic Biology}. DOI: \href{https://doi.org/10.1021/acssynbio.3c00583}{10.1021/acssynbio.3c00583}

  \item \textbf{Arévalo, S.; Pérez Rico, D.; Abarca, D.; Dijkhuizen, L.W.; et al.} (2022-09-19)  
    “Genome engineering by RNA-guided transposition for \textit{Anabaena} PCC 7120,” Preprint.  
    DOI: \href{https://doi.org/10.1101/2022.09.18.508393}{10.1101/2022.09.18.508393}

  \item \textbf{Güngör, E.; Brouwer, P.; Dijkhuizen, L.W.; et al.} (2021-01)  
    “Azolla ferns testify: seed plants and ferns share a common ancestor…,”  
    \emph{New Phytologist}. DOI: \href{https://doi.org/10.1111/nph.16896}{10.1111/nph.16896}

  \item \textbf{Dijkhuizen, L.W.; Güngör, E.; et al.} (2020-07-24)  
    “LAR phylogeny for Gungor et al. 2020: The complete analysis and dataset,” Dataset.  
    DOI: \href{https://doi.org/10.5281/zenodo.3959057}{10.5281/zenodo.3959057}

\end{itemize}

% PhD Thesis Chapters
\noindent
PhD Thesis chapters

\begin{itemize}
  \setlength{\itemsep}{0.3em}
  \item \textbf{Chapter 1: A hitch-hiker’s guide to Azolla symbiosis genomics} \\
    A broad, less formal introduction to Azolla symbiosis genomics, aimed at engaging a wider scientific audience and providing context for the thesis. \\
    \emph{Laura W. Dijkhuizen}
    % No DOI for this chapter

  \item \textbf{Chapter 2: Foul play in the leaf pocket? The metagenome of floating fern Azolla reveals endophytes that do not fix N\textsubscript{2} but may denitrify} \\
    Discovery and analysis of prokaryotic DNA in Azolla, identification of associated bacterial genomes, and investigation of their metabolic pathways and ecological roles. \\
    \emph{Laura W. Dijkhuizen, et al.}
    \href{https://doi.org/10.1111/nph.14843}{DOI}

  \item \textbf{Chapter 3: Hidden treasures: public sequencing data of symbiotic Azolla ferns harbours a genus-wide metagenome} \\
    Development of a workflow to enrich and study genomes of bacteria associated with all sequenced Azolla species, revealing systematic presence and vertical transfer of key symbionts. \\
    \emph{Laura W. Dijkhuizen, et al.}
    % No DOI for this chapter

  \item \textbf{Appendix B: Metagenomics practical} \\
    An educational practical designed to teach metagenomics principles and techniques to Life Sciences students, using Bash and Jupyter notebooks for hands-on learning. \\
    \emph{Laura W. Dijkhuizen}
    \href{https://github.com/lauralwd/metagenomicspractical}{GitHub}

  \item \textbf{Chapter 4: Forever together: One Nostoc azollae is symbiont to all Azolla species} \\
    Comparative genomics of the main Azolla symbiont, N. azollae, showing near-identical genomes across hosts, high pseudogene content, and phylogenomic placement within Nostocales. \\
    \emph{Laura W. Dijkhuizen, et al.}
    % No DOI for this chapter

  \item \textbf{Chapter 5: It takes two: Far-Red light induces the Azolla-Nostoc symbiosis sexual reproduction} \\
    Investigation of sexual reproduction and symbiont transmission in Azolla/N. azollae, including environmental triggers, gene regulation, and evolutionary implications for crop application. \\
    \emph{Laura W. Dijkhuizen, Tabatabaei, B.E.S., Brouwer, P., et al.}
    \href{https://doi.org/10.1101/2020.09.09.289736}{DOI}

  \item \textbf{Chapter 6: One, Two, Tree! A workflow for creating state-of-the-art phylogenies designed for reproducibility with JuPyter, conda and git} \\
    Description of a reproducible workflow for phylogenetic tree inference in land plants, using open-source tools and providing resources for semi-automatic tree annotation. \\
    \emph{Laura W. Dijkhuizen}
    % No DOI for this chapter
\end{itemize}

\end{document}
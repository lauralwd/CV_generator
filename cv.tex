% cv.tex
%—————————————————————————————————––
% Simple asymmetric two-column CV using vars.tex for overrides
%—————————————————————————————————––
\documentclass[a4paper,10pt]{article}
\usepackage[vmargin=1.5cm, hmargin=1.5cm]{geometry}

% Load general setup (fonts, spacing, tikz, adjustbox, etc.)
\input{MySetup}

% Load branch-specific variables
% vars.tex
% ———————––
% Overrideable variables for CV

%––––––––––––––––––––
% Personal details
\newcommand{\FullName}{Laura W. Dijkhuizen}
%\newcommand{\BirthDate}{}
%\newcommand{\BirthPlace}{}
\newcommand{\Residence}{Utrecht, the Netherlands}

\newcommand{\EmailAddress}{laura.w.dijkhuizen@pm.me}
\newcommand{\WebPage}{https://lauradijkhuizen.com}

%––––––––––––––––––––
% File paths
\newcommand{\ProfilePhoto}{pf.jpg}
%\newcommand{\BackgroundImage}{bg-image.pdf}% e.g. a faint watermark or logo

%––––––––––––––––––––
% Style colors (hex RGB without #)
\definecolor{ForestGreen}{HTML}{14532D}


% Go Dutch
\usepackage[dutch]{babel}

% Define a custom gray color in your preamble (or near the top of your document)
\definecolor{BracketGray}{HTML}{888888}

%—————————————————————————————————––
\begin{document}
\pagestyle{fancy}
\thispagestyle{empty}

% setup column nr 1 —————————————————————————————————––
\noindent\makebox[\textwidth][s]{
\begin{adjustbox}{valign=t}
\noindent\begin{minipage}[t]{0.3\textwidth}

\begin{center}
  \begin{tikzpicture}
      \clip (0,0) ellipse (2cm and 2.5cm);
      \node {\includegraphics[width=4cm]{\ProfilePhoto}};
  \end{tikzpicture}

{\begin{spacing}{.8}
  \LARGE \headerfont \FullName
\end{spacing}}

\MySkip

  \ifdefined\BirthDate Born on \BirthDate\\ \fi
  \ifdefined\BirthPlace \BirthPlace\\ \fi
  \ifdefined\Residence \Residence\\ \fi

\MySkip
\textcolor{ForestGreen}{\faEnvelope} \href{mailto:\EmailAddress}{\EmailAddress}\\
\textcolor{ForestGreen}{\faGlobe} \href{\WebPage}{\WebPage}
\end{center}

% About me — R&D data & reporting (tailored)

{\RaggedRight
  Teamspeler met passie voor data, infrastructuur en maatschappelijke impact. 
  Combineert technische vernieuwing met coachend leiderschap en strategisch inzicht.
}

\GreenHeading{Kwaliteiten}
\begin{itemize}[leftmargin=1em, itemindent=0em]
  \item \textbf{Coachend leider:}\\ 
    Biedt structuur waar nodig en stimuleert eigenaarschap in multidisciplinaire teams.
  \item \textbf{Technisch onderlegd verbinder:} \\ 
    Ervaring met FAIR data, workflows, dashboards en infrastructuur, mét oog voor gebruikers en samenwerking.
  \item \textbf{Strategisch denker:} \\ 
    Verbindt lange termijn doelen aan concrete stappen, met gevoel voor beleid, proces en uitvoering.
  \item \textbf{Ervaren procesbegeleider:} \\
    Leidende rol in onderwijsontwikkeling, beleidsadvies en samenwerking over afdelingen en belangen heen.
  \item \textbf{Maatschappelijk gedreven:} \\ 
    Werkt graag aan projecten met publieke waarde — van open wetenschap tot klimaat, gezondheid en infrastructuur.
  \item \textbf{Veranderkracht:} \\
    Ervaring met het begeleiden van verandering op team- en organisatieniveau, met oog voor dynamiek, draagvlak en tempo.
  \item \textbf{Systemisch denker:} \\
    Overziet het geheel en verbindt strategische doelen met praktische uitvoering.
\end{itemize}

% Close column 1 put in vertical line as column2 —————————————————————————————————––

\end{minipage}%
\end{adjustbox}%
\hfill%


\begin{adjustbox}{valign=t}
\hfill%
\begin{minipage}[t]{0.05\textwidth}
\MyVerticalRule
\end{minipage}%
\end{adjustbox}
% Close column 2 and setup column nr 3 —————————————————————————————————––

\begin{adjustbox}{valign=t}
\hfill%
\begin{minipage}[t]{0.6\textwidth}

% Contents column 3 —————————————————————————————————––

% Employment
\GreenHeading{Werkervaring \& Opleiding}
\begin{description}
  \item[\textcolor{ForestGreen}{\textbf{2022 -- heden}}] 
  \textbf{Trainer \& curriculumontwerper PhD-onderwijs:}  
  Coördineert data-analyse training voor promovendi: van inhoudelijke opzet tot afstemming met beleid, stakeholders en docenten.  
  Initieerde een strategisch vijfjarenplan voor AI-ready onderwijs en begeleidt jaarlijks 200 deelnemers in hands-on trainingen.  
  \textit{Universiteit Utrecht}

  \item[\textcolor{ForestGreen}{\textbf{2017 -- 2022}}] 
  \textbf{Promovendus en docent:}  
  Ontwierp en beheerde een FAIR data-infrastructuur voor (meta)genomics-onderzoek.  
  Bood begeleiding aan MSc/BSc-studenten en initieerde standaardisatie en big data binnen het lab.  
  Werkte samen met PPPs zoals Plantum en KeyGene.  
  (\href{https://github.com/lauralwd/azolla_phd_thesis}{GitHub})\\
  \textit{Universiteit Utrecht}
  \item[\textcolor{ForestGreen}{\textbf{2010 -- 2017}}] 
  \textbf{MSc / BSc} Environmental Plant Biology \\
  \textit{Universiteit Utrecht}
\end{description}

% Management
\GreenHeading{Teamleiding, Beleid \& Advies}
\begin{description}

  \item[\textcolor{ForestGreen}{\textbf{2017 -- 2021}}] 
  \textbf{Voorzitter PhD-Council} en lid van de \textbf{Board of Studies} (GS-LS).  
  Vertegenwoordigde 2.000+ promovendi richting decanen, HR en beleidsmedewerkers. 
  Leidde een team van 20 PhD’s en bracht structuur in beleidsoverleg en governance.

  \item[\textcolor{ForestGreen}{\textbf{2019 -- heden}}] 
  Lid van meerdere \textbf{strategische commissies} en overleggremia binnen de Universiteit Utrecht, met focus op Open Science, reproducibility en curriculumontwerp:
  \begin{itemize}
    \item \textbf{Department Advisory Committee} (Biologie)
    \item \textbf{Faculty Open Science Implementation Team}
    \item \textbf{UU Open Science Platform} — brug tussen beleid en praktijk (\href{https://www.uu.nl/en/news/meet-laura-dijkhuizen}{Interview})
    \item \textbf{Curriculumcommissie MSc Bioinformatics \& Biocomplexity} — afgestemd op FAIR en data-intensief onderzoek
  \end{itemize}

  \item[\textcolor{ForestGreen}{\textbf{2021 -- 2022}}] 
  \textbf{Externe adviseur TU Delft}  
  Lid van onafhankelijke visitatiecommissie voor evaluatie van strategie, prestaties en promotiebeleid.  
  Bijdragen vormden input voor senior management.  
  (\href{https://filelist.tudelft.nl/TUDelft/Onderzoek/Kwaliteitsborging/Final report SEP Chemistry TU Delft 20220204.pdf}{Rapport})

  % \item[\textcolor{ForestGreen}{\textbf{2019}}] 
  % \textbf{Interim-leider van het lab}  
  % Tijdens afwezigheid van de PI verantwoordelijk voor planning, voortgang en begeleiding van 6 teamleden.

\end{description}


% Key Achievements & Metrics
\GreenHeading{Geselecteerde Resultaten}
\begin{description}
  \item Ontwierp en implementeerde trainingen in R, Python, datavisualisatie en data governance voor promovendi en postdocs.  
  Aangepast op uiteenlopende achtergronden, waaronder zorg, labonderzoek en milieuonderzoek (gemiddelde waardering 8+/10, 500+ deelnemers).

  \item Bouwde en beheerde een eigen data-infrastructuur voor bioinformatica onderzoek.  
  Dit platform stelde 30+ onderzoekers in staat om reproduceerbaar te werken en data-inzichten te delen met stakeholders.

  \item Ontwikkelde interactieve dashboards en visualisaties voor genetische en microbiële data.  

  \item Initieerde een interfacultaire werkgroep rond promovendi welzijn, voortgang en promotie.
    Dit resulteerde in een nieuw governancekader dat inmiddels 3.500+ promovendi heeft bereikt.

  % \item Adviseerde de TU Delft als lid van een onafhankelijke commissie over de onderzoeksstrategie, prestaties en beleid van twee bèta-afdelingen.

  \item Bouwde verschillende data pipelines van ruwe data tot dashboard en inzicht.
    Nanopore variant calling (\href{https://github.com/lauralwd/anabaena_nanopore_workflow}{link})  
    , pangenomics (\href{https://github.com/lauralwd/Nostoc_azollae_pangenomics}{link})
    , phylogeny (\href{https://github.com/lauralwd/lauras_phylogeny_wf}{link})
    , and metagenome analysis (\href{https://github.com/lauralwd/Azolla_genus_meta_pangenome}{link}). 
\end{description}

% Contents page 2 —————————————————————————————————––
\end{minipage}%
\end{adjustbox}%
}

\newpage

% Teaching & Training
\GreenHeading{Onderwijs \& Training}
\begin{description}
  \item[\textcolor{ForestGreen}{\textbf{2022 -- heden}}] 
  \textbf{Docent \& curriculumontwikkelaar}  
  Hoofdverantwoordelijk voor ontwerp en uitvoering van PhD-cursusaanbod in R, Python, Bash, datavisualisatie en HPC.  
  Nadruk op het belang van basis vaardigheden, reproduceerbaarheid en communicatie van data-inzichten.  
  \item[\textcolor{ForestGreen}{\textbf{2020 -- 2021}}] 
  \textbf{Cursuscoördinator “Intro to Bioinformatics”}  
  Bracht 8+ experts en hun best practices van over de hele uithof samen in één hands-on mastercursus afgestemd op studenten uit uiteenlopende domeinen.  

  (\href{https://lauralwd.github.io/metagenomicspractical/}{GitHub})
  \item[\textcolor{ForestGreen}{\textbf{2017 -- 2022}}] 
  \textbf{Begeleider van MSc/BSc scripties}  
  Begeleiding op gebied van data-analyse, experimenteel ontwerp en visualisatie. Aandacht voor data logica en verantwoording.
\end{description}

\vfill

% Outreach
\GreenHeading{Outreach \& Communicatie}
\begin{description}
  \item \textbf{Televisie:} Interview over \textit{Azolla} varens op lokale omroep (2017)  
    en in het programma “De Kennis van Nu” (NTR, 2018).  
    -- \href{https://youtu.be/OI4VV4M2-f4}{Bekijk} / \href{https://ntr.nl/Focus/287/detail/Onkruid-als-reddende-engel/VPWON_1292624}{Bekijk}
  \item \textbf{Radio:} Interview op BNR Nieuwsradio over de wetenschappelijke relevantie van \textit{Azolla} (2017)  
    -- \href{https://www.bnr.nl/podcast/wetenschap-vandaag/10346708/utrechts-plantje-geniet-wereldwijde-faam}{Beluister}
  \item \textbf{Krantenartikel:} Interview en artikel in het Algemeen Dagblad over eiwitpotentie van kroosvarens (2018)  
    -- \href{https://www.ad.nl/utrecht/kroosachtig-plantje-uit-sloot-naast-galgenwaard-blijkt-ware-eiwitbom~a1eaba6d/}{Lees}
  \item \textbf{Publiekslezingen en demonstraties:}  
    Actieve spreker op open dagen, publiekslezingen en scholen over genetica, symbiose en datawetenschap (2018–2022)
  \item \textbf{Beleids- \& data-overleggen:}  
    Regelmatige deelnemer aan nationale bijeenkomsten over Open Science, data.overheid.nl, en strategisch onderwijsbeleid (VSNU)
  % \item \textbf{Open Science \& reproducibility:}  
  %   Medevormgever van het UU Open Science Platform en het Beta Faculty Open Science Team.
  %   \href{https://www.uu.nl/en/news/meet-laura-dijkhuizen}{Interview op uu.nl}
\end{description}

% SCientific skills and papers
\GreenHeading{Selectie van Wetenschappelijke Publicaties}

\noindent Volledige lijst: \href{https://lauradijkhuizen.com/science}{lauradijkhuizen.com/science} en
\noindent ORCID: \textcolor[HTML]{A6CE39}{\faOrcid}\href{https://orcid.org/0000-0002-4628-7671}{0000-0002-4628-7671}
% \\\\
% \noindent Samenwerkingen met o.a. Jena, Bielefeld, Heinrich Heine University (Duitsland), NIOZ (Nederland), Barcelona (Spanje), Isfahan (Iran), RPI (VS), Boyce Thompson Institute (VS), en meerdere onderzoeksgroepen aan de Universiteit Utrecht.
\begin{description}
  \setlength{\itemsep}{0em}

  \item Studie naar mogelijke \textbf{denitrificatie} als bijeffect van \textbf{stikstoffixatie} in \emph{Azolla}: 
    Is there foul play in the leaf pocket? \\
    {\footnotesize\href{https://doi.org/10.1111/nph.14843}{\emph{New Phytologist} (2018).}}

  \item Een voorbeeld van mijn reproduceerbare fylogenie workflow. Ik publiceerde de \textbf{volledige analyse en dataset}: Azolla ferns testify. \\
    {\footnotesize \href{https://doi.org/10.1111/nph.16896}{\emph{New Phytologist} (2021).}  en
    \href{https://doi.org/10.5281/zenodo.3959057}{Dataset (2020).}}

  \item Een \textbf{plant-insect-microbioom} interactie in een natuurlijke symbiose: Cornicinine uit langpootmuggen triggert cyanobiont-differentiatie in Azolla.\\
    {\footnotesize\href{https://doi.org/10.1111/pce.14907}{\emph{Plant, Cell \& Environment} (2024).}    \null}
  
  \item Verborgen microbiele schatten in \textbf{openbare sequencing data} van symbiotische \emph{Azolla} varens. \\
    {\footnotesize Laura W. Dijkhuizen, et al. thesis chapter}
    % No DOI for this chapter
\end{description}

\vfill

% Interests & Volunteering
\GreenHeading{Interesses \& Vrijwilligerswerk}
\begin{description}
  \raggedright
  \item \textbf{Vrijwilliger en facilitator} van maandelijkse discussiegroepen en begeleider tijdens meerdaagse kampen voor jongeren over gender en inclusie.
  % \item \textbf{Wetenschapscommunicatie \& outreach}: Ervaren spreker op conferenties en publieksbijeenkomsten. Medeorganisator van open science workshops en spreker op onderwijsinitiatieven binnen de universiteit.
  \item \textbf{Actief buitenmens}: Klimmen, fietsen en zeilen — vaak ook met collega’s.
  % \item \textbf{Bèta-dag coördinator}: Organiseerde evenementen voor 150+ middelbare scholieren om kennis te maken met wetenschap en onderzoek.
  \item \textbf{Fotografie}: Gevorderd natuurfotograaf met publicatie op de cover van \href{https://lauralwd.github.io/photography/}{\emph{Nature Plants} (link)}.
  \item \textbf{Groene vingers}: Enthousiast tuinierder en plantenliefhebber.
  \item \textbf{maker} Raspberry PI, sensor netwerken, 3d printers en automatiseering.
\end{description}
\null

\vfill

% Skills\& Languages
\noindent
\begin{minipage}[t]{0.6\textwidth}
    \GreenHeading{Technische en data vaardigheden}
    \begin{tabular}{p{1em}p{11em}p{14.5em}r}
      \textcolor{ForestGreen}{\faServer}   & HPC \& Workflow Ontwerp  & \textcolor{BracketGray}{(Conda, Snakemake)}                   & \SkillBull{$\bullet\bullet\bullet\bullet\bullet$} \\
      \textcolor{ForestGreen}{\faDatabase} & SQL / Data Governance   & \textcolor{BracketGray}{(SQL, iRODS, FAIR)}                    & \SkillBull{$\bullet\bullet\bullet\circ\circ$} \\
      \textcolor{ForestGreen}{\faRProject} & R / Python / Bash       & \textcolor{BracketGray}{(Geavanceerde scripts, Sysadmin)}       & \SkillBull{$\bullet\bullet\bullet\bullet\bullet$} \\
      \textcolor{ForestGreen}{\faDocker}   & Docker / GitHub         & \textcolor{BracketGray}{(Actions, CI/CD)}                     & \SkillBull{$\bullet\bullet\bullet\bullet\circ$} \\
      \textcolor{ForestGreen}{\faChartBar} & Data Dashboards         & \textcolor{BracketGray}{(ggplot2, Shiny, Plotly)}              & \SkillBull{$\bullet\bullet\bullet\bullet\bullet$} \\
      \textcolor{ForestGreen}{\faBook}     & Reproducible Notebooks  & \textcolor{BracketGray}{(Jupyter, Quarto)}                     & \SkillBull{$\bullet\bullet\bullet\bullet\bullet$} \\
    \end{tabular}
\end{minipage}
\hfill
\noindent
\begin{minipage}[t]{.3\textwidth}
\GreenHeading{Talen}
\begin{tabular}{p{1em}p{4em}r}
  \textcolor{ForestGreen}{\faLanguage} & Engels & \SkillBull{$\bullet\bullet\bullet\bullet\bullet$} \\
  \textcolor{ForestGreen}{\faLanguage} & Nederlands   & \SkillBull{$\bullet\bullet\bullet\bullet\bullet$} \\
\end{tabular}
\end{minipage}

\end{document}
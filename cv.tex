% cv.tex
%—————————————————————————————————––
% Simple asymmetric two-column CV using vars.tex for overrides
%—————————————————————————————————––
\documentclass[a4paper,10pt]{article}
\usepackage[vmargin=1.5cm, hmargin=1.5cm]{geometry}

% Load general setup (fonts, spacing, tikz, adjustbox, etc.)
\input{MySetup}

% Load branch-specific variables
% vars.tex
% ———————––
% Overrideable variables for CV

%––––––––––––––––––––
% Personal details
\newcommand{\FullName}{Laura W. Dijkhuizen}
%\newcommand{\BirthDate}{}
%\newcommand{\BirthPlace}{}
\newcommand{\Residence}{Utrecht, the Netherlands}

\newcommand{\EmailAddress}{laura.w.dijkhuizen@pm.me}
\newcommand{\WebPage}{https://lauradijkhuizen.com}

%––––––––––––––––––––
% File paths
\newcommand{\ProfilePhoto}{pf.jpg}
\newcommand{\BackgroundImage}{bg-image.pdf}% e.g. a faint watermark or logo

%––––––––––––––––––––
% Style colors (hex RGB without #)
\definecolor{ColorOne}{HTML}{003399}   % Primary accent color
\definecolor{ColorTwo}{HTML}{FF6600}   % Secondary accent color

%––––––––––––––––––––
% Icon definitions (if needed)
% \newcommand{\IconEmail}{\faEnvelopeO}
% \newcommand{\IconLink}{\faChain}

%––––––––––––––––––––
% Any other branch-specific overrides go here

%—————————————————————————————————––
\begin{document}
\pagestyle{empty}

% setup column nr 1 —————————————————————————————————––
\noindent\makebox[\textwidth][s]{%
\begin{adjustbox}{valign=t}
\noindent\begin{minipage}[t]{0.3\textwidth}

\begin{center}
  \begin{tikzpicture}
      \clip (0,0) ellipse (2cm and 2.5cm);
      \node {\includegraphics[width=4cm]{\ProfilePhoto}};
  \end{tikzpicture}

{\begin{spacing}{.8}
  \LARGE \bfseries \FullName
\end{spacing}}

\MySkip

\ifdefined\BirthDate Born on \BirthDate\\ \fi
\ifdefined\BirthPlace \BirthPlace\\ \fi
\ifdefined\Residence Currently living in \Residence\\ \fi



\MySkip

\textcolor{ColorTwo}{\faEnvelope} \myhref{mailto:\EmailAddress}{\EmailAddress}\\
\textcolor{ColorTwo}{\faGlobe} \myhref{\WebPage}{\WebPage}
\end{center}

\vfill

% About me
{\fontfamily{pag}\selectfont\RaggedRight
  Passionate about making complex data science accessible.
  I bring a strong background in plant biology, bioinformatics, and team coordination. 
  Skilled in stakeholder communication and cross-functional collaboration, 
  I lead high-impact projects with a focus on reproducibility, innovation, and real-world application. 
  I thrive in purpose-driven, autonomous roles where training and tooling empower others.
}
\vfill

% Education
\section*{Education}
\begin{description}
\raggedright
\item[\normalfont \textcolor{ColorOne}{2022.}] Basic Teaching Qualification (BKO), Utrecht University
\item[\normalfont \textcolor{ColorOne}{2017 -- 2022.}] \textbf{PhD: The Azolla metagenome} \\
Utrecht University – Molecular Plant Physiology
\item[\normalfont \textcolor{ColorOne}{2010 -- 2017.}] \textbf{MSc / BSc}\\
\end{description}

\vfill
% Close column 1 put in vertical line as column2 —————————————————————————————————––

\end{minipage}%
\end{adjustbox}%
\hfill%


\begin{adjustbox}{valign=t}
\hfill%
\begin{minipage}[t]{0.05\textwidth}
\MyVerticalRule
\end{minipage}%
\end{adjustbox}
% Close column 2 and setup column nr 3 —————————————————————————————————––

\begin{adjustbox}{valign=t}
\hfill%
\begin{minipage}[t]{0.6\textwidth}

% Contents column 3 —————————————————————————————————––

% Employment
\section*{Employment}
\begin{description}
\raggedright
\item[\normalfont \textcolor{ColorOne}{2022 -- now.}] Lecturer\&  Trainer in programming, bioinformatics \& data science for PhD candidates \& postdocs at Utrecht University – Theoretical Biology \& Bioinformatics Group. Developed \& delivered courses such as \myhref{https://github.com/lauralwd/metagenomicspractical}{“Intro to Bioinformatics”}.
\item[\normalfont \textcolor{ColorOne}{2017 -- 2022.}] PhD Researcher, Plant Physiology Group, Utrecht University.
\end{description}
% Detailed career note: While lab leader was abroad, led the “Azolla” research team—supervising students and guiding an interdisciplinary group to meet project milestones.
% (Original Azolla chapter; see portfolio at \myhref{https://lauradijkhuizen.com}{lauradijkhuizen.com})
\end{minipage}%
\end{adjustbox}%
}
% Contents page 2 —————————————————————————————————––
\newpage

% Publications
\section*{Publications}
\begin{itemize}
\item \underline{Name, Y.}, Second Author, N., Third Author, N. et al. \textbf{(2019)} A great title {\it Journal}. doi:~\myhref{https://doi.org/xx.yyy/zzz.12345}{xx.yyy/zzz.12345}
\item First Author, T., \underline{Name, Y.}, Third Author, N. et al. \textbf{(2015)} A long title for a great study{\it Journal}. doi:~\myhref{https://doi.org/xx.yyy/zzz.54321}{xx.yyy/zzz.54321}
\end{itemize}

% Teaching
\section*{Teaching \& Training}
\begin{description}
\raggedright
\item[\normalfont \textcolor{ColorOne}{2022 -- now.}] Lead trainer for PhD \& postdoc workshops in R, Bash \& Python: customizing modules to individual research goals and ensuring hands-on mastery.
  % Hint: trains across departments—from child oncology to RIVM scientists.
\item[\normalfont \textcolor{ColorOne}{2020 -- 2021.}] Coordinated “\textbf{Introduction to Bioinformatics}” course uniting experts from all Utrecht Science Park institutions into one cohesive curriculum.
  % Hint: aligned domain experts in genomics, structural biology, and data analysis.
\item[\normalfont \textcolor{ColorOne}{2017 -- 2022.}] Supervised MSc/BSc theses and served as TA in Systems Biology, Plant Physiology \& Molecular Genetics.
\item Qualified university teacher (BKO), specialized in adult and professional education.
\end{description}

% Management
\section*{Leadership \& Project Management}
\begin{description}
\raggedright
\item[\normalfont \textcolor{ColorOne}{2023 -- now.}] Member, Department Advisory Committee, Biology Dept., Utrecht University — shaping strategic priorities\& governance.
\item[\normalfont \textcolor{ColorOne}{2021.}] External member, TU Delft Research Assessment Committee (Chemical Engineering\& Biotechnology).
\item[\normalfont \textcolor{ColorOne}{2020 -- 2021.}] Member, Faculty Open Science Implementation Team\& UU Open Science Platform — driving reproducibility policies.
\item[\normalfont \textcolor{ColorOne}{2019 -- 2021.}] Chair, PhD Council (GSLS) — led ~1,800 PhD representatives; negotiated new graduation guidelines.
\item[\normalfont \textcolor{ColorOne}{2019 -- 2021.}] Member, Curriculum Committee, M.Sc. Bioinformatics\& Biocomplexity.
\item[\normalfont \textcolor{ColorOne}{2017 -- 2021.}] PhD Council Representative, Institute of Environmental Biology\& GSLS.
\end{description}
% Hint: occasional leadership of Azolla R&D team; guided interdisciplinary labs and committees while mentoring junior researchers.

% Open Science
\section*{Open Science}
\begin{description}
\raggedright
\item Promotion of open data and open source tools in research projects
\item Development of reproducible workflows for metagenomic analyses
\end{description}

% Outreach
\section*{Outreach\& Communication}
\begin{description}
\raggedright
\item Delivered public lectures (elementary school to RIVM), print\& broadcast interviews, media features.
  % Hint: praised for patience and clarity in explaining complex topics in plain English.
\item Organized and participated in campus-wide science events—“Bèta-dag” science day, open-science seminars.
\end{description}
% Hint: strong network across UU, RIVM, oncology, and industry partners; excels at stakeholder communication.


% Core Competencies
\section*{Core Competencies}
\begin{itemize}
  \item \textbf{Languages \& Scripting:} R, Python, Bash
  \item \textbf{Data Analysis \& Visualization:} Bioconductor, Pandas, Matplotlib, ggplot2
  \item \textbf{High-Performance Computing:} SLURM, Conda, JupyterHub
  \item \textbf{Workflow Management:} Nextflow, Snakemake, custom pipelines
  \item \textbf{Reproducibility \& Collaboration:} Git/GitHub, Docker, Open Science Framework
  \item \textbf{Cloud \& Automation:} AWS S3, GitHub Actions
\end{itemize}

% Selected Projects
\section*{Selected Projects}
\begin{itemize}
  \item \textbf{Intro to Bioinformatics Course (2022)} – Unified experts from Utrecht Science Park into a cohesive curriculum. \myhref{https://github.com/lauralwd/metagenomicspractical}{github.com/lauralwd/metagenomicspractical}
  \item \textbf{HPC Docker Playground (2021)} – Designed a containerized HPC environment with SLURM and user quotas on a Synology NAS.
  \item \textbf{Open Phylogeny Workflow (2020)} – Developed a reproducible Nextflow/JupyterHub pipeline for teaching phylogenetics. \myhref{https://github.com/lauralwd/lauras_phylogeny_wf}{github.com/lauralwd/lauras\_phylogeny\_wf}
\end{itemize}

% Professional Attributes
\section*{Professional Attributes}
\begin{itemize}
  \item Strategic thinker \& consensus builder
  \item Excellent communicator across technical and non-technical audiences
  \item Highly autonomous multitasker with strong project oversight
  \item Passion for open, reproducible, and reliable research practices
\end{itemize}

% Key Achievements & Metrics
\section*{Key Achievements}
\begin{itemize}
  \item Trained over 200 PhD candidates \& postdocs in R, Bash \& Python through customized bioinformatics workshops (2022--now).
  \item Coordinated the “Intro to Bioinformatics” course uniting 8+ domain experts from Utrecht Science Park into a cohesive curriculum.
  \item Reduced data analysis run-time by 40\% by designing a Docker--SLURM HPC playground on a Synology NAS (2021).
  \item Built and maintained 3 reproducible pipelines (Nanopore variant calling, phylogeny workflow, metagenome analysis) now used by 10+ researchers.
  \item Chaired a council of $\sim$1,800 PhD candidates, negotiating and implementing new graduation guidelines adopted university-wide (2019--2021).
  \item Served on 5+ strategic committees (Faculty Open Science, Department Advisory, TU Delft RAC), influencing policy affecting 3 faculties and 2,500+ stakeholders.
  \item Led a cross-functional lab team (3 postdocs \& 5 students) in the Azolla project, resulting in 2 peer-reviewed preprints and a reproducible data package.
  \item Delivered 10+ public talks (elementary schools to RIVM), reaching audiences of 20--200 people.
  \item Featured in 5 media outlets (national newspaper, radio \& TV), explaining complex science to $\sim$100,000 combined listeners/readers.
\end{itemize}

\MySkip

% Skills\& Languages
\begin{multicols}{2}
\section*{Technical Proficiencies}
\begin{tabular}{ll}
  R       & \SkillBull{$\bullet\bullet\bullet\bullet$} \\
  Python  & \SkillBull{$\bullet\bullet\bullet\bullet$} \\
  Bash    & \SkillBull{$\bullet\bullet\bullet\circ$} \\
\end{tabular}

\vfill\null \columnbreak

\section*{Languages \& Fluency}
\begin{tabular}{ll}
  English & Fluent \\
  Dutch   & Fluent \\
\end{tabular}

\end{multicols}

% Skill Visualization
\section*{Skill Visualization}
\begin{description}
  \item[\faServer\ HPC]          \SkillBar{0.80}{4cm}
  \item[\faDocker\ Docker]        \SkillBar{1.00}{4cm}
  \item[\faGithub\ Git]           \SkillBar{0.90}{4cm}
  \item[\faPuzzlePiece\ Snakemake] \SkillBar{0.90}{4cm} \Tag{Snakemake}
  \item[\faGithub\ GitHub Actions] \SkillBar{0.80}{4cm} \Tag{CI/CD}
  \item[\faDatabase\ SQL]         \SkillBar{0.50}{4cm}
  \item[\faBrain\ ML]             \SkillBar{0.60}{4cm}
  \item[\faChartBar\ Data Vis]    \SkillBar{0.90}{4cm}
 % \item[\faTachometer\ Dashboards] \SkillBar{0.85}{4cm}
  \item[\faBook\ Reporting]       \SkillBar{0.90}{4cm} \Tag{Notebooks}
\end{description}

% Last update command from MySetup
\LastUpdate

\end{document}
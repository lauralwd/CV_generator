% cv.tex
%—————————————————————————————————––
% Simple asymmetric two-column CV using vars.tex for overrides
%—————————————————————————————————––
\documentclass[a4paper,10pt]{article}
\usepackage[vmargin=1.5cm, hmargin=1.5cm]{geometry}

% Load general setup (fonts, spacing, tikz, adjustbox, etc.)
\input{MySetup}
% Load branch-specific variables
% vars.tex
% ———————––
% Overrideable variables for CV

%––––––––––––––––––––
% Personal details
\newcommand{\FullName}{Laura W. Dijkhuizen}
%\newcommand{\BirthDate}{}
%\newcommand{\BirthPlace}{}
\newcommand{\Residence}{Utrecht, the Netherlands}

\newcommand{\EmailAddress}{laura.w.dijkhuizen@pm.me}
\newcommand{\WebPage}{https://lauradijkhuizen.com}

%––––––––––––––––––––
% File paths
\newcommand{\ProfilePhoto}{pf.jpg}
\newcommand{\BackgroundImage}{bg-image.pdf}% e.g. a faint watermark or logo

%––––––––––––––––––––
% Style colors (hex RGB without #)
\definecolor{ColorOne}{HTML}{003399}   % Primary accent color
\definecolor{ColorTwo}{HTML}{FF6600}   % Secondary accent color

%––––––––––––––––––––
% Icon definitions (if needed)
% \newcommand{\IconEmail}{\faEnvelopeO}
% \newcommand{\IconLink}{\faChain}

%––––––––––––––––––––
% Any other branch-specific overrides go here


%—————————————————————————————————––
\begin{document}
\pagestyle{empty}

\noindent\makebox[\textwidth][s]{%
\begin{adjustbox}{valign=t}
\noindent\begin{minipage}[t]{0.3\textwidth}
\begin{center}
  \begin{tikzpicture}
    \clip (0,0) circle (2cm) node {\includegraphics[width=3cm]{\ProfilePhoto}};
  \end{tikzpicture}

\MySkip

{\LARGE \bfseries \FullName}

\MySkip

Born on \BirthDate\\
\BirthPlace\\
Currently living in \Residence\\

\MySkip

\textcolor{ColorTwo}{\faEnvelopeO} \myhref{mailto:\EmailAddress}{\EmailAddress}\\
\textcolor{ColorTwo}{\faChain} \myhref{\WebPage}{\WebPage}
\end{center}

\vfill

% About me
\section*{About me}
\raggedright
Friends describe me as a versatile young academic and teacher, often contributing to committees and panels throughout the Utrecht University organisation. 
My research interests lie in plant physiology, symbioses, evolutionary- and meta-genomics, and ecology. 
I specialise in using my computational skills to answer appealing biological questions in those specific areas.
Outside of the university, you may find me rock climbing, tinkering with electronics, or photographing the beautiful city of Utrecht. A more elaborate portfolio of my professional and hobby activities can be found on my (slightly outdated) personal website.

\vfill

% Education
\section*{Education}
\begin{description}
\raggedright
\item[\normalfont \textcolor{ColorOne}{2022.}] Basic Teaching Qualification (BKO), Utrecht University
\item[\normalfont \textcolor{ColorOne}{2017 -- 2022.}] \textbf{PhD: The Azolla metagenome} \\
Utrecht University – Molecular Plant Physiology
\item[\normalfont \textcolor{ColorOne}{2010 -- 2017.}] \textbf{MSc / BSc}\\
\end{description}

\vfill
\end{minipage}%
\end{adjustbox}%
\hfill%


\begin{adjustbox}{valign=t}
\hfill%
\begin{minipage}[t]{0.05\textwidth}
\MyVerticalRule
\end{minipage}%
\end{adjustbox}


\begin{adjustbox}{valign=t}
\hfill%
\begin{minipage}[t]{0.6\textwidth}
% Employment
\section*{Employment}
\begin{description}
\raggedright
\item[\normalfont \textcolor{ColorOne}{2022 -- now.}] Lecturer\&  Trainer in programming, bioinformatics \& data science for PhD candidates \& postdocs at Utrecht University – Theoretical Biology \& Bioinformatics Group. Developed \& delivered courses such as \myhref{https://github.com/lauralwd/metagenomicspractical}{“Intro to Bioinformatics”}.
\item[\normalfont \textcolor{ColorOne}{2017 -- 2022.}] PhD Researcher, Plant Physiology Group, Utrecht University.
\end{description}
% Detailed career note: While lab leader was abroad, led the “Azolla” research team—supervising students and guiding an interdisciplinary group to meet project milestones.
% (Original Azolla chapter; see portfolio at \myhref{https://lauradijkhuizen.com}{lauradijkhuizen.com})
\end{minipage}%
\end{adjustbox}%
}

\newpage

% Publications
\section*{Publications}
\begin{itemize}
\item \underline{Name, Y.}, Second Author, N., Third Author, N. et al. \textbf{(2019)} A great title {\it Journal}. doi:~\myhref{https://doi.org/xx.yyy/zzz.12345}{xx.yyy/zzz.12345}
\item First Author, T., \underline{Name, Y.}, Third Author, N. et al. \textbf{(2015)} A long title for a great study{\it Journal}. doi:~\myhref{https://doi.org/xx.yyy/zzz.54321}{xx.yyy/zzz.54321}
\end{itemize}

% Teaching
\section*{Teaching}
\begin{description}
\raggedright
\item[\normalfont \textcolor{ColorOne}{2022 -- now.}] Lecturer and course developer for bioinformatics trainings tailored to PhD candidates and postdocs in Life Sciences. I teach programming, data analysis and visualization using R, Bash, and Python, with a strong emphasis on reproducibility and real-world data.
\item[\normalfont \textcolor{ColorOne}{2020 -- 2021.}] Course coordinator for “Introduction to Bioinformatics” at the Master's level. Designed new modules and assignments, some of which are available on GitHub.
\item[\normalfont \textcolor{ColorOne}{2018 -- 2021.}] Teacher in “Biology of the Bio-Based Economy” and “Introduction to Bioinformatics” for MSc students. Focused on hands-on computational work.
\item[\normalfont \textcolor{ColorOne}{2017 -- 2022.}] Supervision of MSc and BSc thesis students in computational biology and plant physiology projects.
\item[\normalfont \textcolor{ColorOne}{2017 -- 2020.}] Teaching assistant in courses such as Systems Biology, Plant Physiology, and Molecular Genetics.
\item Qualified university lecturer (BKO). Specialised in educating early-career researchers and professionals.
\end{description}

% Management
\section*{Leadership, Management \& Committees}
\begin{description}
\raggedright
\item[\normalfont \textcolor{ColorOne}{2023 -- now.}] Member of the Department Advisory Committee for Biology, Utrecht University.
\item[\normalfont \textcolor{ColorOne}{2021.}] External committee member for the TU Delft Research Assessment Committee for Chemical Engineering and Biotechnology departments.
\item[\normalfont \textcolor{ColorOne}{2020 -- 2021.}] Member of the Faculty Open Science Implementation Team and the Utrecht University Open Science Platform.
\item[\normalfont \textcolor{ColorOne}{2019 -- 2021.}] Chair of the PhD Council at the Graduate School of Life Sciences (GSLS). Represented PhD voices in the Board of Studies.
\item[\normalfont \textcolor{ColorOne}{2019 -- 2021.}] Member of the curriculum committee for the Master's in Bioinformatics \& Biocomplexity.
\item[\normalfont \textcolor{ColorOne}{2017 -- 2021.}] PhD council representative at the Institute of Environmental Biology and the GSLS.
\item Active in organising educational events, feedback sessions, and curriculum improvements across multiple layers of the university.
\end{description}

% Open Science
\section*{Open Science}
\begin{description}
\raggedright
\item Promotion of open data and open source tools in research projects
\item Development of reproducible workflows for metagenomic analyses
\end{description}

% Outreach
\section*{Outreach}
\begin{description}
\raggedright
\item Photography of Utrecht and surrounding nature for public engagement
\item Participation in science communication events and activities
\end{description}

\MySkip

% Skills & Languages
\begin{multicols}{2}
\section*{Numerical tools}
\begin{tabular}{ll}
R              & \SkillBull{$\bullet\bullet\bullet\,\circ$} \\
Matlab         & \SkillBull{$\bullet\bullet\bullet\,\circ$} \\
Mathematica    & \SkillBull{$\bullet\,\circ\,\circ\,\circ$} \\
\end{tabular}

\vfill\null \columnbreak

\section*{Languages}
\begin{tabular}{ll}
  Spanish        & \SkillBull{$\bullet \bullet \bullet \, \circ$}\\
  French         & \SkillBull{$\bullet \, \circ \, \circ \, \circ$}\\
\end{tabular}

\end{multicols}

% Last update command from MySetup
\LastUpdate

\end{document}
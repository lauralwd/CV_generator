% cv.tex
%—————————————————————————————————––
% Simple asymmetric two-column CV using vars.tex for overrides
%—————————————————————————————————––
\documentclass[a4paper,10pt]{article}
\usepackage[vmargin=1.5cm, hmargin=1.5cm]{geometry}

% Load general setup (fonts, spacing, tikz, adjustbox, etc.)
\input{MySetup}

% Load branch-specific variables
% vars.tex
% ———————––
% Overrideable variables for CV

%––––––––––––––––––––
% Personal details
\newcommand{\FullName}{Laura W. Dijkhuizen}
%\newcommand{\BirthDate}{}
%\newcommand{\BirthPlace}{}
\newcommand{\Residence}{Utrecht, the Netherlands}

\newcommand{\EmailAddress}{laura.w.dijkhuizen@pm.me}
\newcommand{\WebPage}{https://lauradijkhuizen.com}

%––––––––––––––––––––
% File paths
\newcommand{\ProfilePhoto}{pf.jpg}
\newcommand{\BackgroundImage}{bg-image.pdf}% e.g. a faint watermark or logo

%––––––––––––––––––––
% Style colors (hex RGB without #)
\definecolor{ColorOne}{HTML}{003399}   % Primary accent color
\definecolor{ColorTwo}{HTML}{FF6600}   % Secondary accent color

%––––––––––––––––––––
% Icon definitions (if needed)
% \newcommand{\IconEmail}{\faEnvelopeO}
% \newcommand{\IconLink}{\faChain}

%––––––––––––––––––––
% Any other branch-specific overrides go here

% Define a custom gray color in your preamble (or near the top of your document)
\definecolor{BracketGray}{HTML}{888888}

%—————————————————————————————————––
\begin{document}
\pagestyle{fancy}
\thispagestyle{empty}

% setup column nr 1 —————————————————————————————————––
\noindent\makebox[\textwidth][s]{
\begin{adjustbox}{valign=t}
\noindent\begin{minipage}[t]{0.3\textwidth}

\begin{center}
  \begin{tikzpicture}
      \clip (0,0) ellipse (2cm and 2.5cm);
      \node {\includegraphics[width=4cm]{\ProfilePhoto}};
  \end{tikzpicture}

{\begin{spacing}{.8}
  \LARGE \headerfont \FullName
\end{spacing}}

\MySkip

  \ifdefined\BirthDate Born on \BirthDate\\ \fi
  \ifdefined\BirthPlace \BirthPlace\\ \fi
  \ifdefined\Residence \Residence\\ \fi

\MySkip
\textcolor{ForestGreen}{\faEnvelope} \href{mailto:\EmailAddress}{\EmailAddress}\\
\textcolor{ForestGreen}{\faGlobe} \href{\WebPage}{\WebPage}
\end{center}

% About me — R&D data & reporting (tailored)

{\RaggedRight
  Ik ben een \textbf{bioloog}, \textbf{data-educator}, en verbindend junior \textbf{leider} klaar voor de volgende stap.  \\
  Ik ben op mijn plek in wetenschap-gedreven teams die werken aan complexe maatschappelijke en milieuvraagstukken.
}

\GreenHeading{Kwaliteiten}
\begin{itemize}[leftmargin=1em, itemindent=0em]
  \setlength{\itemsep}{0.0em}
  \item \textbf{Mensgerichte leider:}\\ 
    Behulpzaam en benaderbaar;  
    vertegenwoordigde 2.000+ promovendi en betrokken in allerlei beleidsorganen.
  \item \textbf{Strategisch systeemdenker:} \\ 
    Ontwerp bioinformatica curriculum voor professionals (8+/10, 500+ deelnemers).
  \item \textbf{Interdisciplinair verbinder:} \\ 
    Projectcoördinatie 8+ Interdisciplinaire experts. 
  \item \textbf{Heldere communicator:} \\ 
    Schakel tussen PhD-council en bestuur; 
    ervaren spreker in wetenschappelijke en maatschappelijke contexten.
  \item \textbf{Bouwt reproduceerbare datasystemen:} \\ 
    Ontwikkeling FAIR workflows en dashboards voor 30+ onderzoekers; 
  \item \textbf{Vertrouwde adviseur in beleid:} \\ 
    Advies TU Delft over R\&D-strategie en PhD beleid;
    verbindend in bestuurlijke omgevingen en netwerken.
\end{itemize}

% Close column 1 put in vertical line as column2 —————————————————————————————————––

\end{minipage}%
\end{adjustbox}%
\hfill%


\begin{adjustbox}{valign=t}
\hfill%
\begin{minipage}[t]{0.05\textwidth}
\MyVerticalRule
\end{minipage}%
\end{adjustbox}
% Close column 2 and setup column nr 3 —————————————————————————————————––

\begin{adjustbox}{valign=t}
\hfill%
\begin{minipage}[t]{0.6\textwidth}

% Contents column 3 —————————————————————————————————––

% Employment
\GreenHeading{Werkervaring \& Opleiding}
\begin{description}
\raggedright
  \item[\normalfont \textcolor{ForestGreen}{\textbf{2022 -- heden.}}] \textbf{Ontwikkelaar \& Trainer} 
    van promovendi curriculum voor programmeren, bioinformatica \& datawetenschap 
    (\href{https://github.com/lauralwd/professional_education}{GitHub})\\
    \textit{Theoretische Biologie \& Bioinformatica, Universiteit Utrecht}
  \item[\normalfont \textcolor{ForestGreen}{\textbf{2017 -- 2022.}}] \textbf{Promovendus en docent} 
    Ik verzekerde €250k subsidie voor mijn PhD over de (meta)genomics van de \textit{Azolla} symbiose
    (\href{https://github.com/lauralwd/azolla_phd_thesis}{GitHub}) 
    % Cofinanciering en samenwerking met Plantum, KeyGene en B-Ware.\\ % skip because niche?
    \textit{Moleculaire Plantenfysiologie, Universiteit Utrecht}
  \item[\normalfont \textcolor{ForestGreen}{\textbf{2010 -- 2017.}}] \textbf{MSc / BSc} Environmental Plant Biology \\
    \textit{Universiteit Utrecht}
\end{description}

% Management
\GreenHeading{Leiderschap \& Teamcoördinatie}

% \emph{Coordinated 8+ expert instructors; chaired council representing >2,000 PhDs; frequent liaison with deans/HR/graduate schools to drive policy and adoption.}
\begin{description}
\raggedright
  \item[\normalfont \textcolor{ForestGreen}{\textbf{2021 -- 2022.}}] 
    Lid van \textbf{onafhankelijk R\&D beoordelingspanel} voor de TU Delft voor evaluatie van strategie, prestaties en beleid 
    binnen de afdelingen Chemische Technologie en Biotechnologie. 
    Adviezen vormden input voor het senior management. 
    (\href{https://filelist.tudelft.nl/TUDelft/Onderzoek/Kwaliteitsborging/Final report SEP Chemistry TU Delft 20220204.pdf}{Rapport}).
  
  \item[\normalfont \textcolor{ForestGreen}{\textbf{2019.}}] \textbf{Interim-leider} 
    van het lab tijdens afwezigheid van de PI.
    Verantwoordelijk voor doorlopend onderzoek, planning, en begeleiding van teamleden en stagaires.
 
  \item[\normalfont \textcolor{ForestGreen}{\textbf{2019 -- heden.}}] 
    Lid van meerdere strategische commissies binnen Universiteit Utrecht;
    bijdragen aan beleidsontwikkeling, reproducibility en curriculumontwerp:
    \begin{itemize}
      \item \textbf{Department Advisory Committee} (Biologie)
      \item \textbf{Faculty Open Science Implementation Team} \& \textbf{UU Open Science Platform} 
      (\href{https://www.uu.nl/en/news/meet-laura-dijkhuizen}{Interview}).
      \item \textbf{Curriculum Committee} voor de MSc Bioinformatics \& Biocomplexity 
        (\href{https://www.uu.nl/en/masters/bioinformatics-and-biocomplexity}{link}).
    \end{itemize}
  \item[\normalfont \textcolor{ForestGreen}{\textbf{2017 -- 2021.}}] \textbf{Voorzitter PhD-Council} en lid van de \textbf{Board of Studies} (GS-LS).
    Ik vertegenwoordigde 2.000+ promovendi; stelde prioriteiten, coördineerde en delegeerde taken, en trok een lijn voor het team om te volgen.
\end{description}


% Key Achievements & Metrics
\GreenHeading{Geselecteerde Resultaten}
\begin{description}
  \item Ontwierp en implementeerde een bioinformatica-curriculum vanaf nul met oog op consistente en kwalitatieve data analyse en raportage.
    Laura wordt beoordeeld met gemiddeld een 8+/10 door 500+ deelnemers.
    Het eerste 'MVP' werd binnen enkele weken opgeleverd en wordt nog steeds succesvol gegeven.
  \item Bouwde een gedeeld data- en analyseplatform met FAIR-structuur, role-based toegang en geïntegreerde notebooks.
    Dit platform stelde 30+ onderzoekers in staat om reproduceerbaar te werken en data-inzichten te delen.
  \item Initieerde een interfacultaire werkgroep rond promovendi welzijn, voortgang en promotie.
    Dit resulteerde in een nieuw governancekader voor promovendi dat inmiddels >3.500 promovendi bereikt.
  %\item Taught “Intro to Bioinformatics” course; cohesively uniting various domain experts from Utrecht Science Park.
  \item Bouwde verschillende data pipelines van ruwe data tot dashboard en inzicht.
    Nanopore variant calling (\href{https://github.com/lauralwd/anabaena_nanopore_workflow}{link})  
    , pangenomics (\href{https://github.com/lauralwd/Nostoc_azollae_pangenomics}{link})
    , phylogeny (\href{https://github.com/lauralwd/lauras_phylogeny_wf}{link})
    , and metagenome analysis (\href{https://github.com/lauralwd/Azolla_genus_meta_pangenome}{link}). 
\end{description}

% Contents page 2 —————————————————————————————————––
\end{minipage}%
\end{adjustbox}%
}

\newpage

% Teaching & Training
\GreenHeading{Onderwijs \& Training}
\begin{description}
  \raggedright
  \item[\normalfont \textcolor{ForestGreen}{\textbf{2022 -- heden}}] \textbf{Docent \& trainer}: 
    Ontwikkeling en uitvoering van cursussen in R, Python en Bash voor promovendi en postdocs. 
    Aandacht voor data-governance en communicatie van data-inzichten richting verschillende belanghebbenden.  
    (\href{https://github.com/lauralwd/professional_education}{GitHub})
  \item[\normalfont \textcolor{ForestGreen}{\textbf{2020 -- 2021}}] \textbf{Cursuscoördinator “Intro to Bioinformatics”}: 
    Bracht 8+ experts samen in één hands-on mastercursus, afgestemd op de behoeften van een breed deelnemersveld. 
    (\href{https://lauralwd.github.io/metagenomicspractical/}{GitHub})
  \item[\normalfont \textcolor{ForestGreen}{\textbf{2017 -- 2022}}] \textbf{Begeleider van MSc/BSc scripties}: 
    Ondersteunde studenten bij experimenteel ontwerp, data-analyse en visuele rapportage van resultaten.
\end{description}

\vfill


% SCientific skills and papers
\GreenHeading{Selectie van Wetenschappelijke Publicaties}

\noindent Volledige lijst: \href{https://lauradijkhuizen.com/science}{lauradijkhuizen.com/science} en
\noindent ORCID: \textcolor[HTML]{A6CE39}{\faOrcid}\href{https://orcid.org/0000-0002-4628-7671}{0000-0002-4628-7671}
\\\\
\noindent Samenwerkingen met o.a. Jena, Bielefeld, Heinrich Heine University (Duitsland), NIOZ (Nederland), Barcelona (Spanje), Isfahan (Iran), RPI (VS), Boyce Thompson Institute (VS), en meerdere onderzoeksgroepen aan de Universiteit Utrecht.

\vspace{0.5em}

\noindent \textbf{Geselecteerde artikelen:}
\begin{itemize}
  \setlength{\itemsep}{0em}

  \item Studie naar mogelijke \textbf{denitrificatie} als bijeffect van \textbf{stikstoffixatie} in \emph{Azolla}: 
    Is there foul play in the leaf pocket? \\
    {\null\hfill\footnotesize\href{https://doi.org/10.1111/nph.14843}{\emph{New Phytologist} (2018).}}

  \item Een voorbeeld van mijn reproduceerbare fylogenie workflow. Ik publiceerde de \textbf{volledige analyse en dataset}: Azolla ferns testify. 
    {\null\hfill\footnotesize \href{https://doi.org/10.1111/nph.16896}{\emph{New Phytologist} (2021).}  en
    \href{https://doi.org/10.5281/zenodo.3959057}{Dataset (2020).}}

  \item Een \textbf{plant-insect-microbioom} interactie in een natuurlijke symbiose: Cornicinine uit langpootmuggen triggert cyanobiont-differentiatie in Azolla.
    {\null\hfill\footnotesize\href{https://doi.org/10.1111/pce.14907}{\emph{Plant, Cell \& Environment} (2024).} }
\end{itemize}

\vfill

% Interests & Volunteering
\GreenHeading{Interesses \& Vrijwilligerswerk}
\begin{description}
  \raggedright
  \item \textbf{Vrijwilliger en facilitator} van maandelijkse discussiegroepen en begeleider tijdens meerdaagse kampen voor jongeren over gender en inclusie.
  % \item \textbf{Wetenschapscommunicatie \& outreach}: Ervaren spreker op conferenties en publieksbijeenkomsten. Medeorganisator van open science workshops en spreker op onderwijsinitiatieven binnen de universiteit.
  \item \textbf{Actief buitenmens}: Klimmen, fietsen en zeilen — vaak ook met collega’s.
  % \item \textbf{Bèta-dag coördinator}: Organiseerde evenementen voor 150+ middelbare scholieren om kennis te maken met wetenschap en onderzoek.
  \item \textbf{Fotografie}: Gevorderd natuurfotograaf met publicatie op de cover van \href{https://lauralwd.github.io/photography/}{\emph{Nature Plants} (link)}.
  \item \textbf{Groene vingers}: Enthousiast tuinier en plantenliefhebber.
  \item \textbf{maker} Raspberry PI, sensor netwerken, 3d printers en automatiseering.
\end{description}

\vfill

% Skills\& Languages
\noindent
\begin{minipage}[t]{0.6\textwidth}
    \GreenHeading{Technical \& Data Governance Proficiencies}
    \begin{tabular}{p{1em}p{15em}p{5em}r}
      \textcolor{ForestGreen}{\faServer}   & HPC \& Workflow Design         & \SkillBull{$\bullet\bullet\bullet\bullet\bullet$} \\
      \textcolor{ForestGreen}{\faDatabase} & SQL / Data Governance            & \SkillBull{$\bullet\bullet\bullet\bullet\circ$} \\
      \textcolor{ForestGreen}{\faRProject} & R / Python / Bash              & \SkillBull{$\bullet\bullet\bullet\bullet\bullet$} \\
      \textcolor{ForestGreen}{\faDocker}   & Docker / GitHub CI             & \SkillBull{$\bullet\bullet\bullet\bullet\circ$} \\
      \textcolor{ForestGreen}{\faDatabase} & 3D printing, electronic tinkering & \SkillBull{$\bullet\bullet\bullet\circ\circ$} \\
      \textcolor{ForestGreen}{\faChartBar} & Data Dashboards \& Notebooks    & \SkillBull{$\bullet\bullet\bullet\bullet\bullet$} \\
    \end{tabular}
\end{minipage}
\hfill
\noindent
\begin{minipage}[t]{.3\textwidth}
\GreenHeading{Languages}
\begin{tabular}{p{1em}p{4em}r}
  \textcolor{ForestGreen}{\faLanguage} & English & \SkillBull{$\bullet\bullet\bullet\bullet\bullet$} \\
  \textcolor{ForestGreen}{\faLanguage} & Dutch   & \SkillBull{$\bullet\bullet\bullet\bullet\bullet$} \\
\end{tabular}
\end{minipage}

\vfill

\end{document}